\documentclass[hyperref, a4paper]{ctexart}
\usepackage{lmodern}
\usepackage{amssymb,amsmath}
\usepackage{ifxetex,ifluatex}
\usepackage{fixltx2e} % provides \textsubscript
\ifnum 0\ifxetex 1\fi\ifluatex 1\fi=0 % if pdftex
  \usepackage[T1]{fontenc}
  \usepackage[utf8]{inputenc}
\else % if luatex or xelatex
  \ifxetex
    \usepackage{xltxtra,xunicode}
  \else
    \usepackage{fontspec}
  \fi
  \defaultfontfeatures{Mapping=tex-text,Scale=MatchLowercase}
  \newcommand{\euro}{€}
\fi
% use upquote if available, for straight quotes in verbatim environments
\IfFileExists{upquote.sty}{\usepackage{upquote}}{}
% use microtype if available
\IfFileExists{microtype.sty}{%
\usepackage{microtype}
\UseMicrotypeSet[protrusion]{basicmath} % disable protrusion for tt fonts
}{}
\ifxetex
  \usepackage[setpagesize=false, % page size defined by xetex
              unicode=false, % unicode breaks when used with xetex
              xetex]{hyperref}
\else
  \usepackage[unicode=true]{hyperref}
\fi
\usepackage[usenames,dvipsnames]{color}
\hypersetup{breaklinks=true,
            bookmarks=true,
            pdfauthor={Tian, Jiahe; Hu, Xiaoxiao; Huang, Jiani; Liu, Jiaxing; Shi, Ruixin; Wu, Chenning; Zhang, Cenyuan; Zhang, Yihan; Wang, Chen},
            pdftitle={ 性能测试设计与执行},
            colorlinks=true,
            citecolor=blue,
            urlcolor=blue,
            linkcolor=magenta,
            pdfborder={0 0 0}}
\urlstyle{same}  % don't use monospace font for urls
\usepackage{graphicx,grffile}
\makeatletter
\def\maxwidth{\ifdim\Gin@nat@width>\linewidth\linewidth\else\Gin@nat@width\fi}
\def\maxheight{\ifdim\Gin@nat@height>\textheight\textheight\else\Gin@nat@height\fi}
\makeatother
% Scale images if necessary, so that they will not overflow the page
% margins by default, and it is still possible to overwrite the defaults
% using explicit options in \includegraphics[width, height, ...]{}
\setkeys{Gin}{width=\maxwidth,height=\maxheight,keepaspectratio}
\setlength{\emergencystretch}{3em}  % prevent overfull lines
\providecommand{\tightlist}{%
  \setlength{\itemsep}{0pt}\setlength{\parskip}{0pt}}
\setcounter{secnumdepth}{5}

\title{\vspace{2in} 性能测试设计与执行\\\vspace{0.5em}{\large 软件质量保障与测试课程Lab7课程作业(第9组)}}
\author{Tian, Jiahe\footnote{Equal Contribution, Fudan University, 17307130313
  (\href{mailto:tianjh17@fudan.edu.cn}{\nolinkurl{tianjh17@fudan.edu.cn}})} \and Hu, Xiaoxiao\footnote{Equal Contribution, Fudan University, 17302010077
  (\href{mailto:xxhu17@fudan.edu.cn}{\nolinkurl{xxhu17@fudan.edu.cn}})} \and Huang, Jiani\footnote{Equal Contribution, Fudan University, 17302010063
  (\href{mailto:huangjn17@fudan.edu.cn}{\nolinkurl{huangjn17@fudan.edu.cn}})} \and Liu, Jiaxing\footnote{Equal Contribution, Fudan University, 17302010049
  (\href{mailto:jiaxingliu17@fudan.edu.cn}{\nolinkurl{jiaxingliu17@fudan.edu.cn}})} \and Shi, Ruixin\footnote{Equal Contribution, Fudan University, 17302010065
  (\href{mailto:rxshi17@fudan.edu.cn}{\nolinkurl{rxshi17@fudan.edu.cn}})} \and Wu, Chenning\footnote{Equal Contribution, Fudan University, 17302010066
  (\href{mailto:cnwu17@fudan.edu.cn}{\nolinkurl{cnwu17@fudan.edu.cn}})} \and Zhang, Cenyuan\footnote{Equal Contribution, Fudan University,
  17302010068
  (\href{mailto:cenyuanzhang17@fudan.edu.cn}{\nolinkurl{cenyuanzhang17@fudan.edu.cn}})} \and Zhang, Yihan\footnote{Equal Contribution, Fudan University, 17302010076
  (\href{mailto:zhangyihan17@fudan.edu.cn}{\nolinkurl{zhangyihan17@fudan.edu.cn}})} \and Wang, Chen\footnote{Equal Contribution, Fudan University, 16307110064
  (\href{mailto:wangc16@fudan.edu.cn}{\nolinkurl{wangc16@fudan.edu.cn}})}}
\date{2020年5月14日}



% Redefines (sub)paragraphs to behave more like sections
\ifx\paragraph\undefined\else
\let\oldparagraph\paragraph
\renewcommand{\paragraph}[1]{\oldparagraph{#1}\mbox{}}
\fi
\ifx\subparagraph\undefined\else
\let\oldsubparagraph\subparagraph
\renewcommand{\subparagraph}[1]{\oldsubparagraph{#1}\mbox{}}
\fi

\begin{document}
\maketitle

\newpage

\LARGE

\begin{center}
\textbf{性能测试设计与执行}
\end{center}

\large
\begin{center}
\textbf{\emph{软件质量保障与测试课程Lab7课程作业}}
\end{center}

\hypertarget{ux6458ux8981}{%
\section*{摘要}\label{ux6458ux8981}}
\addcontentsline{toc}{section}{摘要}

本次作业为软件质量保障与测试课程的Lab6课程作业,需要我们以小组为单位完成对出题系统的性能测试。本文档分为两小节。第一小节介绍了本小组进行性能测试采用的策略;第二小节介绍了性能测试的结果及系统性能分析。

\hypertarget{ux5173ux952eux8bcd}{%
\section*{关键词}\label{ux5173ux952eux8bcd}}
\addcontentsline{toc}{section}{关键词}

系统与软件工程; 系统与软件质量要求和评价; 测试文档

\normalsize

\newpage

\tableofcontents

\newpage

\hypertarget{sonarux5de5ux5177ux9759ux6001ux6d4bux8bd5}{%
\section{Sonar工具静态测试}\label{sonarux5de5ux5177ux9759ux6001ux6d4bux8bd5}}

\hypertarget{ux6d4bux8bd5ux7ed3ux679cux5b8cux6574ux62a5ux544a}{%
\subsection{测试结果完整报告}\label{ux6d4bux8bd5ux7ed3ux679cux5b8cux6574ux62a5ux544a}}

``出题系统''前端(Javascript)检测报告见附件2020-05-24-test-maker-fore-report
``出题系统''后端(Java)检测报告见附件2020-05-24-my\_project-report

\hypertarget{ux6d4bux8bd5ux7ed3ux679cux6838ux5fc3ux5185ux5bb9ux622aux56fe}{%
\subsection{测试结果核心内容截图}\label{ux6d4bux8bd5ux7ed3ux679cux6838ux5fc3ux5185ux5bb9ux622aux56fe}}

sonar检测报告网页版视图中,分别以总览、问题、安全热点、指标、代码来记录对代码的分析结果,下面将分别截图这些部分。
sonar检测报告的文档版的内容与网页版一致,具体报告在附件中可以查看。

\hypertarget{ux524dux7aefux68c0ux6d4bux62a5ux544a}{%
\subsubsection{前端检测报告}\label{ux524dux7aefux68c0ux6d4bux62a5ux544a}}

\includegraphics{screenshots/lab7-pic/fore-1.png}
\includegraphics{screenshots/lab7-pic/fore-2.png}
\includegraphics{screenshots/lab7-pic/fore-3.png}
\includegraphics{screenshots/lab7-pic/fore-5.png}
\includegraphics{screenshots/lab7-pic/fore-6.png}

\hypertarget{ux540eux7aefux68c0ux6d4bux62a5ux544a}{%
\subsubsection{后端检测报告}\label{ux540eux7aefux68c0ux6d4bux62a5ux544a}}

\includegraphics{screenshots/lab7-pic/back-1.png}
\includegraphics{screenshots/lab7-pic/back-2.png}
\includegraphics{screenshots/lab7-pic/back-3.png}
\includegraphics{screenshots/lab7-pic/back-5.png}
\includegraphics{screenshots/lab7-pic/back-6.png}

\hypertarget{ux6d4bux8bd5ux7ed3ux679cux5206ux6790}{%
\subsection{测试结果分析}\label{ux6d4bux8bd5ux7ed3ux679cux5206ux6790}}

Sonar是一个用于代码质量管理的开源平台,用于管理源代码的质量。

\hypertarget{sonarux6d4bux8bd5ux62a5ux544aux7279ux70b9}{%
\subsubsection{sonar测试报告特点}\label{sonarux6d4bux8bd5ux62a5ux544aux7279ux70b9}}

sonar的code
viewer(代码)部分向开发者展示代码源文件和高层次的数据,包含了行数、问题数、单元覆盖率、重复度、代码近期提交信息等。其中,coverage用三种色彩可视化标记,红色表示没有覆盖,橙色表示部分覆盖,绿色表示完全覆盖。duplications计算重复代码的行数并定位。

sonar的issues(问题)部分分析的粒度为类型、严重程度、处理方式、状态、标准、新问题、语言、规则、标签、目录、文件、负责人、作者。严重程度划分为blocker,critical,major,minor,info,对应致命(阻断),关键(严重),主要,微小(次要),未知(提示)。在issues中,sonar还会列出maintainability,documentation,complexity,bulk
change,dispositioning等项。

\hypertarget{ux626bux63cfux6548ux679c}{%
\subsubsection{扫描效果}\label{ux626bux63cfux6548ux679c}}

选择的是sonar默认的扫描规则,效果非常细致,相对来说扫描过程用时也比较久。sonar将问题分为bud,code\_smell和vulnerability,严重程度划分为minor,major,blocked,critical,info。bug类型的问题多涉及到比如比较判断符错误使用,空指针重定向,条件分支不可达等,code\_smell类型涉及到方法返回值不能一直不变,方法体不能为空等,vulnerability类型涉及引用了已知有漏洞的函数的代码。
sonar扫描的问题比较全面,也可以做到持续的代码检查跟进,具有高可用性和较短的反馈循环。提供了多种语言检测支持。

\hypertarget{p3cux5de5ux5177ux9759ux6001ux6d4bux8bd5}{%
\section{p3c工具静态测试}\label{p3cux5de5ux5177ux9759ux6001ux6d4bux8bd5}}

\hypertarget{ux6d4bux8bd5ux7ed3ux679cux5b8cux6574ux62a5ux544a-1}{%
\subsection{测试结果完整报告}\label{ux6d4bux8bd5ux7ed3ux679cux5b8cux6574ux62a5ux544a-1}}

请通过网址\href{https://straybird-atsh.github.io/SoftwareQA-Testing/P3CReport.html}{第9组P3C报告网站}来查看本小组的P3C完整报告。

\hypertarget{ux6d4bux8bd5ux7ed3ux679cux6838ux5fc3ux5185ux5bb9ux622aux56fe-1}{%
\subsection{测试结果核心内容截图}\label{ux6d4bux8bd5ux7ed3ux679cux6838ux5fc3ux5185ux5bb9ux622aux56fe-1}}

如图所示,可分为Blocker、Critical、Major三部分。
\includegraphics{screenshots/pic1.jpg}
\includegraphics{screenshots/pic2.jpg}

\hypertarget{ux6d4bux8bd5ux7ed3ux679cux5206ux6790-1}{%
\subsection{测试结果分析}\label{ux6d4bux8bd5ux7ed3ux679cux5206ux6790-1}}

P3C是阿里巴巴推出的《阿里巴巴 Java 开发规约》扫描插件,目前在 IDEA 和
Eclipse 都有较好的支持。P3C扫描结果文档给出了项目的 bug,并根据 bug
的严重程度分为三个级别展示。三个级别分别是:Blocker, Critical,
Major,严重程度由高到低。

\hypertarget{ux963fux91ccux5df4ux5df4-java-ux5f00ux53d1ux624bux518c}{%
\subsubsection{阿里巴巴 JAVA
开发手册}\label{ux963fux91ccux5df4ux5df4-java-ux5f00ux53d1ux624bux518c}}

根据《阿里巴巴 JAVA
开发手册》版本1.3.0,我们可以对扫描结果进行分析。该版本对 JAVA
开发的规约含有编程规约、异常日志、单元测试、安全规约等六大类,而本项目的扫描结果重点体现的是编程规约。编程规约在开发手册共分为9类:

\begin{itemize}
\tightlist
\item
  命名规范
\item
  常量定义
\item
  代码格式
\item
  OOP 规约
\item
  集和处理
\item
  并发处理
\item
  控制语句
\item
  注释规约
\item
  其他
\end{itemize}

\hypertarget{ux626bux63cfux6548ux679c-1}{%
\subsubsection{扫描效果}\label{ux626bux63cfux6548ux679c-1}}

根据扫描结果可以发现:同一类型的编程规约由于其严重性不同可以划分为不同级别的
bug。比如注释规约中,抽象方法的 javadoc 注释属于
Major,而枚举字段的注释属于 Critical;
命名规范中也有类似的例子。另外,bug
的严重程度分类比较合理。注释规约、命名规范的约定内容严重程度较低,并发处理、控制语句的约定内容严重程度较高。\textbf{整体上,P3C扫描插件发现的问题比较基础,它侧重
JAVA 编程细节可能导致的系统失效。}

\hypertarget{ux63d2ux4ef6ux4f7fux7528}{%
\subsubsection{插件使用}\label{ux63d2ux4ef6ux4f7fux7528}}

P3C插件有以下的优点:

\begin{itemize}
\tightlist
\item
  基本满足代码规范检测的需求。
\item
  能够检测出细致和易忽略的问题,可以提高开发过程中对代码细枝末节的注意。
\item
  Quick fix,检测出问题后可以快速查看 bug 位置、解决方案并一键替换。
\item
  中文提示,解释内容与开发手册一致。
\end{itemize}

此外P3C插件也存在平台限制、功能不成熟、扫描能力有限等问题。

\hypertarget{jshintux5de5ux5177ux9759ux6001ux6d4bux8bd5}{%
\section{jshint工具静态测试}\label{jshintux5de5ux5177ux9759ux6001ux6d4bux8bd5}}

\hypertarget{ux6d4bux8bd5ux7ed3ux679cux5b8cux6574ux62a5ux544a-2}{%
\subsection{测试结果完整报告}\label{ux6d4bux8bd5ux7ed3ux679cux5b8cux6574ux62a5ux544a-2}}

请通过网址\href{https://straybird-atsh.github.io/SoftwareQA-Testing/JSHintReport.html}{第9组JSHint报告网站}来查看本小组的JSHint完整报告。

\hypertarget{ux6d4bux8bd5ux7ed3ux679cux6838ux5fc3ux5185ux5bb9ux622aux56fe-2}{%
\subsection{测试结果核心内容截图}\label{ux6d4bux8bd5ux7ed3ux679cux6838ux5fc3ux5185ux5bb9ux622aux56fe-2}}

如图所示,共发现236个failures,1个error和235个warnings。其中每一项都标明了具体的行数、问题代码和问题原因。
\includegraphics{screenshots/pic3.jpg}
\includegraphics{screenshots/pic3.jpg}

\hypertarget{ux6d4bux8bd5ux7ed3ux679cux5206ux6790-2}{%
\subsection{测试结果分析}\label{ux6d4bux8bd5ux7ed3ux679cux5206ux6790-2}}

JSHint 是由Anton Kovalyov基于 JSLint 的代码实现的开源项目,JSHint 与
JSLint
相比较之下,更友好,也更容易配置,所以发展很快并得到了众多IDE和编辑器的支持。
JSHint 是一个 JavaScript
语法和风格的检查工具,但检查不出逻辑问题。它可以根据配置参数扫描
JavaScript代码,分析其中的语法与风格从而给出代码质量报告。

\hypertarget{ux914dux7f6eux9879}{%
\subsubsection{配置项}\label{ux914dux7f6eux9879}}

\includegraphics{screenshots/jshintrc.png} JSHint
工具使用的关键是配置项。如果不设置配置项,那么可能会有很多``假''错误或警告。比如自定义全局变量,不同脚本上下文的符号引用。这些内容既不算错误的语法,也不算差劲的风格,可是
JSHint 依旧把他们认错了。JSHint 有一些常见的配置项:

\begin{itemize}
\tightlist
\item
  ``strict'': true 严格模式
\item
  ``asi'': true 允许省略分号
\item
  ``bitwise'': true 禁止使用位运算符,比如经常把\&\&写错\& 规避此错误
\item
  ``eqeqeq'': true 禁止使用== 和 != 强制使用=== 和 !==
\item
  ``undef'': true 禁止使用不在全局变量列表中的未定义变量
\item
  ``curly'': true 循环或者条件语句必须使用花括号包住
\item
  ``devel'': true 定义用于调试的全局变量:console,alert
\item
  ``jquery'': true 定义全局暴露的jQuery库 (可以去掉)
\item
  ``browser'': true 暴露浏览器属性的全局变量 如window document
\item
  ``globals'': \{``\$'':true,``require'':true,"\_":true\}
\end{itemize}

JSHint 的配置项一般是放入项目根目录下的 .jshintrc 文件中,JSHint
工具在扫描的时候就会运用这些配置项。配置项的作用一方面是明确项目的编程规范,约束开发人员的行为。另一方面是避免某些规范,减轻开发人员的负担(比如允许省略分号等)。

\hypertarget{ux626bux63cfux6548ux679c-2}{%
\subsubsection{扫描效果}\label{ux626bux63cfux6548ux679c-2}}

不同的配置项扫描结果是不尽相同的。项目最初未定义配置项中的全局变量,这导致了非常多的未定义错误或警告。这显然不是我们需要的扫描结果。完成配置项后,扫描结果报告逐渐明晰,其中报告了出现了缺少分号、使用==和!=错误、缺少\{或\}错误和未定义变量等错误。正如上面所言,JSHint
可以分析出 JavaScript
代码的语法和风格,但是它无法识别出项目的逻辑错误,比如冗余代码、死循环、无效代码等问题。它的分析功能是非常基础和有限的。

\hypertarget{ux5de5ux5177ux4f7fux7528}{%
\subsubsection{工具使用}\label{ux5de5ux5177ux4f7fux7528}}

JSHint 工具不可否认具有一定的优点:

\begin{itemize}
\tightlist
\item
  有了很多参数可以配置
\item
  支持配置文件,方便使用
\item
  支持了一些常用类库(如jquery等)
\item
  支持了基本的ES6
\end{itemize}

但与其他功能强大的 Web
项目静态扫描工具比较而言,它具有不支持自定义规则、无法根据错误定位到对应的规则和不提供快捷的修正方式等缺点。不支持自定义规则自然让它只能检测出很基础的预定义规则,无法根据错误定位到对应的规则使得扫描结果不易阅览,不提供快捷的修正方式(不能跳转到指定代码位置和不能提供修正方案)自然无法让开发人员方便的处理扫描结果。

\hypertarget{ux4e0dux540cux5de5ux5177ux4e4bux95f4ux7684ux5bf9ux6bd4ux5206ux6790}{%
\section{不同工具之间的对比分析}\label{ux4e0dux540cux5de5ux5177ux4e4bux95f4ux7684ux5bf9ux6bd4ux5206ux6790}}

\begin{itemize}
\item
  sonar:定位是代码质量平台,本身不进行代码分析,但可以集成各个静态分析工具以及其他软件开发测试工具,并基于集成工具的结果数据按照一定的质量模型,对软件的质量进行评估。甚至可以选择接入p3c规范进行代码扫描。sonar基于扫描规则进行扫描,因此扫描的问题可以比较全面,本身是代码质量平台,可以做到持续的代码检查跟进,具有高可用性和较短的反馈循环。提供了多种语言检测支持。生成的检测报告也比较详细,数据可视化好。是几个工具中专业性最高的。
\item
  p3c是一套自动化的IDE检测插件(主要是IDEA、Eclipse)该插件在扫描代码后,将不符合《手册》的代码按Blocker/Critical/Major三个等级显示在下方,根据错误定位到对应的规则,在IDE中提供快捷修正方式,但p3c扫描检测的问题较为基础,侧重JAVA编程细节可能导致的系统失效,之后仍需要类似FindBugs的插件再次扫描检测bug。
\item
  JSHint 是由Anton Kovalyov 基于JSLint
  的代码实现的开源项目,是一个JavaScript
  语法和风格检查的命令行工具,不能检查出逻辑问题。它可以根据配置参数扫描JavaScript代码,
  分析其中的语法风格从而给出代码质量报告。相比于sonar它不支持自定义规则、相比于p3c它无法根据错误定位到对应的规则,也不提供快捷的修正方式,分析功能也非常基础和有限,是并不算强大的web静态扫描工具。
\end{itemize}

\hypertarget{ux6d4bux8bd5ux603bux7ed3}{%
\section{测试总结}\label{ux6d4bux8bd5ux603bux7ed3}}

在我们选取的三个测试工具中,sonar定位于代码质量平台,集成各种静态分析工具和其它测试工具,提供持续代码检测跟进,是最强大和专业的静态测试工具。而p3c和JSHint分别是针对Java和JS开发的基于编码及设计实践的静态检测工具,p3c做成了IDE检测插件,JSHint则是命令行工具,由于二者都偏重于代码编写格式,及是否符合编码规范的检验,内置编程规范比较基础,对代码
bug发现功能较弱。
但是他们都能快速定位代码隐藏错误和缺陷,显著减少在代码逐行检查上花费的时间,提高软件可靠性并节省软件开发和测试成本。

\pagebreak

\hypertarget{ux53c2ux8003ux6587ux732e}{%
\section*{参考文献}\label{ux53c2ux8003ux6587ux732e}}
\addcontentsline{toc}{section}{参考文献}

\hypertarget{refs}{}
\leavevmode\hypertarget{ref-innovativeInternationalisation}{}%
International Organization for Standardization. 2014. \emph{Systems and
Software Engineering --- Systems and Software Quality Requirements and
Evaluation (SQuaRE) --- Guide to SQuaRE}. \emph{International
Organization for Standardization}. Vol. 2014.
\url{https://www.iso.org/standard/64764.html}.

\leavevmode\hypertarget{ref-innovative1}{}%
中国国家标准化管理委员会. 2016. \emph{GB/T
25000.51-2016《系统与软件工程系统与软件质量要求和评价 (SQuaRE) 第 51
部分 : 就绪可用软件产品 (RUSP) 的质量要求和测试细则》}.
\emph{系统与软件工程系统与软件质量要求和评价 (SQuaRE)}. Vol. 51.
中国国家标准化管理委员会. \url{http://openstd.samr.gov.cn}.

\leavevmode\hypertarget{ref-innovative3}{}%
---------. 2017a. \emph{GB/T 25000.12-2017《系统与软件工程
系统与软件质量要求和评价(SQuaRE) 第12部分:数据质量模型》}.
\emph{系统与软件工程系统与软件质量要求和评价 (SQuaRE)}. Vol. 12.
中国国家标准化管理委员会. \url{http://openstd.samr.gov.cn}.

\leavevmode\hypertarget{ref-innovative4}{}%
---------. 2017b. \emph{GB/T 25000.24-2017《系统与软件工程
系统与软件质量要求和评价(SQuaRE) 第24部分:数据质量测量》}.
\emph{系统与软件工程系统与软件质量要求和评价 (SQuaRE)}. Vol. 24.
中国国家标准化管理委员会. \url{http://openstd.samr.gov.cn}.

\leavevmode\hypertarget{ref-innovative5}{}%
---------. 2018. \emph{GB/T 25000.40-201《系统与软件工程
系统与软件质量要求和评价(SQuaRE) 第40部分:评价过程》}.
\emph{系统与软件工程系统与软件质量要求和评价 (SQuaRE)}. Vol. 40.
中国国家标准化管理委员会. \url{http://openstd.samr.gov.cn}.

\leavevmode\hypertarget{ref-innovative2}{}%
---------. 2019. \emph{GB/T 25000.23-2019《系统与软件工程
系统与软件质量要求和评价(SQuaRE) 第23部分:系统与软件产品质量测量》}.
\emph{系统与软件工程系统与软件质量要求和评价 (SQuaRE)}. Vol. 23.
中国国家标准化管理委员会. \url{http://openstd.samr.gov.cn}.

\end{document}
