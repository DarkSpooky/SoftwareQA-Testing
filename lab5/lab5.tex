\documentclass[hyperref, a4paper]{ctexart}
\usepackage{lmodern}
\usepackage{amssymb,amsmath}
\usepackage{ifxetex,ifluatex}
\usepackage{fixltx2e} % provides \textsubscript
\ifnum 0\ifxetex 1\fi\ifluatex 1\fi=0 % if pdftex
  \usepackage[T1]{fontenc}
  \usepackage[utf8]{inputenc}
\else % if luatex or xelatex
  \ifxetex
    \usepackage{xltxtra,xunicode}
  \else
    \usepackage{fontspec}
  \fi
  \defaultfontfeatures{Mapping=tex-text,Scale=MatchLowercase}
  \newcommand{\euro}{€}
\fi
% use upquote if available, for straight quotes in verbatim environments
\IfFileExists{upquote.sty}{\usepackage{upquote}}{}
% use microtype if available
\IfFileExists{microtype.sty}{%
\usepackage{microtype}
\UseMicrotypeSet[protrusion]{basicmath} % disable protrusion for tt fonts
}{}
\ifxetex
  \usepackage[setpagesize=false, % page size defined by xetex
              unicode=false, % unicode breaks when used with xetex
              xetex]{hyperref}
\else
  \usepackage[unicode=true]{hyperref}
\fi
\usepackage[usenames,dvipsnames]{color}
\hypersetup{breaklinks=true,
            bookmarks=true,
            pdfauthor={Tian, Jiahe; Hu, Xiaoxiao; Huang, Jiani; Liu, Jiaxing; Shi, Ruixin; Wu, Chenning; Zhang, Cenyuan; Zhang, Yihan; Wang, Chen},
            pdftitle={ 黑盒测试设计与执行},
            colorlinks=true,
            citecolor=blue,
            urlcolor=blue,
            linkcolor=magenta,
            pdfborder={0 0 0}}
\urlstyle{same}  % don't use monospace font for urls
\setlength{\emergencystretch}{3em}  % prevent overfull lines
\providecommand{\tightlist}{%
  \setlength{\itemsep}{0pt}\setlength{\parskip}{0pt}}
\setcounter{secnumdepth}{5}

\title{\vspace{2in} 黑盒测试设计与执行\\\vspace{0.5em}{\large 软件质量保障与测试课程Lab5课程作业(第9组)}}
\author{Tian, Jiahe\footnote{Equal Contribution, Fudan University, 17307130313
  (\href{mailto:tianjh17@fudan.edu.cn}{\nolinkurl{tianjh17@fudan.edu.cn}})} \and Hu, Xiaoxiao\footnote{Equal Contribution, Fudan University, 17302010077
  (\href{mailto:xxhu17@fudan.edu.cn}{\nolinkurl{xxhu17@fudan.edu.cn}})} \and Huang, Jiani\footnote{Equal Contribution, Fudan University, 17302010063
  (\href{mailto:huangjn17@fudan.edu.cn}{\nolinkurl{huangjn17@fudan.edu.cn}})} \and Liu, Jiaxing\footnote{Equal Contribution, Fudan University, 17302010049
  (\href{mailto:jiaxingliu17@fudan.edu.cn}{\nolinkurl{jiaxingliu17@fudan.edu.cn}})} \and Shi, Ruixin\footnote{Equal Contribution, Fudan University, 17302010065
  (\href{mailto:rxshi17@fudan.edu.cn}{\nolinkurl{rxshi17@fudan.edu.cn}})} \and Wu, Chenning\footnote{Equal Contribution, Fudan University, 17302010066
  (\href{mailto:cnwu17@fudan.edu.cn}{\nolinkurl{cnwu17@fudan.edu.cn}})} \and Zhang, Cenyuan\footnote{Equal Contribution, Fudan University,
  17302010068
  (\href{mailto:cenyuanzhang17@fudan.edu.cn}{\nolinkurl{cenyuanzhang17@fudan.edu.cn}})} \and Zhang, Yihan\footnote{Equal Contribution, Fudan University, 17302010076
  (\href{mailto:zhangyihan17@fudan.edu.cn}{\nolinkurl{zhangyihan17@fudan.edu.cn}})} \and Wang, Chen\footnote{Equal Contribution, Fudan University, 16307110064
  (\href{mailto:wangc16@fudan.edu.cn}{\nolinkurl{wangc16@fudan.edu.cn}})}}
\date{2020年5月2日}



% Redefines (sub)paragraphs to behave more like sections
\ifx\paragraph\undefined\else
\let\oldparagraph\paragraph
\renewcommand{\paragraph}[1]{\oldparagraph{#1}\mbox{}}
\fi
\ifx\subparagraph\undefined\else
\let\oldsubparagraph\subparagraph
\renewcommand{\subparagraph}[1]{\oldsubparagraph{#1}\mbox{}}
\fi

\begin{document}
\maketitle

\newpage

\LARGE

\begin{center}
\textbf{黑盒测试设计与执行}
\end{center}

\large
\begin{center}
\textbf{\emph{软件质量保障与测试课程Lab5课程作业}}
\end{center}

\hypertarget{ux6458ux8981}{%
\section*{摘要}\label{ux6458ux8981}}
\addcontentsline{toc}{section}{摘要}

本次作业为软件质量保障与测试课程的Lab5课程作业,需要我们以小组为单位撰写在线出题考试系统的IEEE829测试文档.

\hypertarget{ux5173ux952eux8bcd}{%
\section*{关键词}\label{ux5173ux952eux8bcd}}
\addcontentsline{toc}{section}{关键词}

系统与软件工程; 系统与软件质量要求和评价; 测试文档

\normalsize

\newpage

\tableofcontents

\newpage

\hypertarget{ux6d4bux8bd5ux6846ux67b6ux8bbeux8ba1}{%
\section{测试框架设计}\label{ux6d4bux8bd5ux6846ux67b6ux8bbeux8ba1}}

\hypertarget{ux9ed1ux76d2ux6d4bux8bd5ux6846ux67b6ux56fe}{%
\subsection{黑盒测试框架图}\label{ux9ed1ux76d2ux6d4bux8bd5ux6846ux67b6ux56fe}}

\hypertarget{ux9ed1ux76d2ux6d4bux8bd5ux6846ux67b6ux56feux8bf4ux660e}{%
\subsection{黑盒测试框架图说明}\label{ux9ed1ux76d2ux6d4bux8bd5ux6846ux67b6ux56feux8bf4ux660e}}

\hypertarget{ux5173ux952eux5b57ux53caux6d4bux8bd5ux6570ux636eux8bbeux8ba1}{%
\section{关键字及测试数据设计}\label{ux5173ux952eux5b57ux53caux6d4bux8bd5ux6570ux636eux8bbeux8ba1}}

\hypertarget{ux6d4bux8bd5ux5173ux952eux5b57}{%
\subsection{测试关键字}\label{ux6d4bux8bd5ux5173ux952eux5b57}}

\hypertarget{ux767bux5f55}{%
\subsubsection{登录}\label{ux767bux5f55}}

\hypertarget{ux521bux5efaux8003ux9898}{%
\subsubsection{创建考题}\label{ux521bux5efaux8003ux9898}}

创建考题需要完成考题的相关设定。这些设定包括选择章节、选择知识点、设置作者、设置评审
、设置质管、设置类型、设置语言、设置出题日期、设置评审日期。而创建考题作为一个独立的UI测试,登录是一个基础环节。于是我们针对上述的关键步骤设计了以下关键字:
+ loginActions 登录系统 + navigateToAddQuestion 页面导航 +
showEditQuestion 选择添加考题 + chooseChapter 选择章节 +
chooseKnowledgePoint 选择知识点 + chooseAuthor 设置作者 + chooseReviewer
设置评审 + chooseQA 设置质管 + chooseType 设置题目类型 + startDate
设置出题开始日期 + finishDate 设置出题结束日期 + reviewStartDate
设置评审开始日期 + reviewFinishDate 设置评审结束日期 + chooseLanguage
选择语言 + saveQuestion 保存考题

其中我们没有将登录拆分为更加细致的行为,这是因为登录对于创建考题环节而言是比较基础和简单的。它不是本黑盒测试的考察内容。

\hypertarget{ux6d4bux8bd5ux6570ux636e}{%
\subsection{测试数据}\label{ux6d4bux8bd5ux6570ux636e}}

\hypertarget{ux767bux5f55-1}{%
\subsubsection{登录}\label{ux767bux5f55-1}}

\#\#\#\#等价类划分与边界值分析

\#\#\#\#测试数据

\hypertarget{ux521bux5efaux8003ux9898-1}{%
\subsubsection{创建考题}\label{ux521bux5efaux8003ux9898-1}}

\#\#\#\#等价类划分与边界值分析
创建考题的UI测试具备多个输入参数,并且这些参数的取值也多种多样。我们对输入参数进行等价类划分和边界值分析。

\textbf{等价类划分}

输入参数

有效等价类

无效等价类

chapter

{[}1{]}任何章节

{[}2{]}为空

Knowledge point

{[}3{]}章节下任意知识点

{[}4{]}为空

Author

{[}5{]}任意用户

{[}6{]}为空

Reviewer

{[}7{]}除author外用户

{[}8{]}author {[}9{]}为空

QA

{[}10{]}除author,reviewer外用户

{[}11{]}author或reviewer {[}12{]}为空

Type

{[}13{]}任意类型

{[}14{]}为空

Start date

{[}15{]}任意日期

{[}16{]}为空

Finish date

{[}17{]}在开始日期及之后的日期

{[}18{]}为空

Review start date

{[}19{]}出题开始日期及之后日期

{[}20{]}为空 {[}21{]}出题开始日期之前

Review finish date

{[}22{]}在评审开始日期及之后的日期,并且不早于出题结束日期

{[}23{]}为空 {[}24{]}出题结束日期之前

language

{[}25{]}任意语言

{[}26{]}为空

边界值分析适用于具有连续取值的参数分析,题目中具有连续取值的只有出题日期与评审日期。
其中出题开始日期不存在边界限定,故不考虑。而对于评审结束日期,它的边界值与评审开始日期和出题结束日期相关。而这两个日期不存在约束关系,
故它的多个边界值条件可以同时成立。

\textbf{边界值分析}

输入参数

边界值

Finish date

{[}27{]}与Start date相同 {[}28{]}Start date后一天

Review start date

{[}29{]}Start date前一天 {[}30{]}与Start date相同 {[}31{]}Start
date后一天

Review finish date

{[}32{]}与Review start date相同 {[}33{]}Review start date后一天
{[}34{]}Finish date前一天 {[}35{]}与Finish date相同 {[}36{]}Finish
date后一天

\#\#\#\#测试数据

\textbf{等价类划分}

ID

覆盖的类

chapter

knowledge point

author

reviewer

QA

type

start date

finish date

review start date

review finish date

language

预期结果

1

1,3,5,7,10,13,15,17,19,22,25

1

1.2.1

testadmin

jmeter022

jmeter023

情景题

2020-05-03

2020-05-03

2020-05-03

2020-05-03

中文

成功

2

1,3,5,7,10,13,15,17,19,22,25

1

1.2.1

jmeter002

jmeter003

jmeter004

视频题

2020-05-03

2020-05-04

2020-05-03

2020-05-04

英文

成功

3

2

空

空

jmeter002

jmeter003

jmeter004

视频题

2020-05-03

2020-05-04

2020-05-03

2020-05-04

英文

失败

4

4

1

空

jmeter002

jmeter003

jmeter004

视频题

2020-05-03

2020-05-04

2020-05-03

2020-05-04

英文

失败

5

6

1

1.2.1

空

jmeter022

jmeter023

情景题

2020-04-29

2020-04-30

2020-04-29

2020-04-30

中文

失败

6

9

1

1.2.1

testadmin

空

jmeter023

情景题

2020-05-03

2020-05-04

2020-05-03

2020-05-04

中文

失败

7

8

1

1.2.1

testadmin

testadmin

jmeter023

情景题

2020-05-03

2020-05-04

2020-05-03

2020-05-04

中文

失败

8

12

1

1.2.1

testadmin

jmeter022

空

情景题

2020-05-03

2020-05-04

2020-05-03

2020-05-04

中文

失败

9

11

1

1.2.1

testadmin

jmeter022

jmeter022

情景题

2020-05-03

2020-05-04

2020-05-03

2020-05-04

中文

失败

10

14

1

1.2.1

testadmin

jmeter022

jmeter023

空

2020-05-03

2020-05-04

2020-05-03

2020-05-04

中文

失败

11

16

1

1.2.1

testadmin

jmeter022

jmeter023

情景题

空

2020-05-04

2020-05-03

2020-05-04

中文

失败

12

18

1

1.2.1

testadmin

jmeter022

jmeter023

情景题

2020-05-03

空

2020-05-03

2020-05-04

中文

失败

13

20

1

1.2.1

testadmin

jmeter022

jmeter023

情景题

2020-05-03

2020-05-04

空

2020-05-04

中文

失败

14

23

1

1.2.1

testadmin

jmeter022

jmeter023

情景题

2020-05-03

2020-05-04

2020-05-03

空

中文

失败

15

26

1

1.2.1

testadmin

jmeter022

jmeter023

情景题

2020-05-03

2020-05-04

2020-05-03

2020-05-04

空

失败

16

21

1

1.2.1

testadmin

jmeter022

jmeter023

情景题

2020-05-03

2020-05-04

2020-04-30

2020-05-04

中文

失败

17

24

1

1.2.1

testadmin

jmeter022

jmeter023

情景题

2020-05-03

2020-05-04

2020-05-03

2020-05-03

中文

失败

\textbf{边界值分析}

ID

覆盖的类

chapter

knowledge point

author

reviewer

QA

type

start date

finish date

review start date

review finish date

language

预期结果

18

27,30,32,35

1

1.2.1

testadmin

jmeter022

jmeter023

情景题

2020-05-03

2020-05-03

2020-05-03

2020-05-03

中文

成功

19

28,31,33,36

1

1.2.1

testadmin

jmeter022

jmeter023

情景题

2020-05-03

2020-05-04

2020-05-04

2020-05-05

中文

成功

20

29

1

1.2.1

testadmin

jmeter022

jmeter023

情景题

2020-05-03

2020-05-04

2020-05-02

2020-05-04

中文

失败

21

34

1

1.2.1

testadmin

jmeter022

jmeter023

情景题

2020-05-03

2020-05-04

2020-05-03

2020-05-03

中文

失败

\hypertarget{ux6d4bux8bd5ux811aux672cux5b9eux73b0ux53caux8fd0ux884c}{%
\section{测试脚本实现及运行}\label{ux6d4bux8bd5ux811aux672cux5b9eux73b0ux53caux8fd0ux884c}}

\hypertarget{ux811aux672cux5b9eux73b0ux65b9ux5f0f}{%
\subsection{脚本实现方式}\label{ux811aux672cux5b9eux73b0ux65b9ux5f0f}}

\hypertarget{ux8fd0ux884cux622aux56feux53caux8bf4ux660e}{%
\subsection{运行截图及说明}\label{ux8fd0ux884cux622aux56feux53caux8bf4ux660e}}

\pagebreak

\hypertarget{ux53c2ux8003ux6587ux732e}{%
\section*{参考文献}\label{ux53c2ux8003ux6587ux732e}}
\addcontentsline{toc}{section}{参考文献}

\hypertarget{refs}{}
\leavevmode\hypertarget{ref-innovativeInternationalisation}{}%
International Organization for Standardization. 2014. \emph{Systems and
Software Engineering --- Systems and Software Quality Requirements and
Evaluation (SQuaRE) --- Guide to SQuaRE}. \emph{International
Organization for Standardization}. Vol. 2014.
\url{https://www.iso.org/standard/64764.html}.

\leavevmode\hypertarget{ref-innovative1}{}%
中国国家标准化管理委员会. 2016. \emph{GB/T
25000.51-2016《系统与软件工程系统与软件质量要求和评价 (SQuaRE) 第 51
部分 : 就绪可用软件产品 (RUSP) 的质量要求和测试细则》}.
\emph{系统与软件工程系统与软件质量要求和评价 (SQuaRE)}. Vol. 51.
中国国家标准化管理委员会. \url{http://openstd.samr.gov.cn}.

\leavevmode\hypertarget{ref-innovative3}{}%
---------. 2017a. \emph{GB/T 25000.12-2017《系统与软件工程
系统与软件质量要求和评价(SQuaRE) 第12部分:数据质量模型》}.
\emph{系统与软件工程系统与软件质量要求和评价 (SQuaRE)}. Vol. 12.
中国国家标准化管理委员会. \url{http://openstd.samr.gov.cn}.

\leavevmode\hypertarget{ref-innovative4}{}%
---------. 2017b. \emph{GB/T 25000.24-2017《系统与软件工程
系统与软件质量要求和评价(SQuaRE) 第24部分:数据质量测量》}.
\emph{系统与软件工程系统与软件质量要求和评价 (SQuaRE)}. Vol. 24.
中国国家标准化管理委员会. \url{http://openstd.samr.gov.cn}.

\leavevmode\hypertarget{ref-innovative5}{}%
---------. 2018. \emph{GB/T 25000.40-201《系统与软件工程
系统与软件质量要求和评价(SQuaRE) 第40部分:评价过程》}.
\emph{系统与软件工程系统与软件质量要求和评价 (SQuaRE)}. Vol. 40.
中国国家标准化管理委员会. \url{http://openstd.samr.gov.cn}.

\leavevmode\hypertarget{ref-innovative2}{}%
---------. 2019. \emph{GB/T 25000.23-2019《系统与软件工程
系统与软件质量要求和评价(SQuaRE) 第23部分:系统与软件产品质量测量》}.
\emph{系统与软件工程系统与软件质量要求和评价 (SQuaRE)}. Vol. 23.
中国国家标准化管理委员会. \url{http://openstd.samr.gov.cn}.

\end{document}
