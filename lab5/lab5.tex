\documentclass[hyperref, a4paper]{ctexart}
\usepackage{lmodern}
\usepackage{amssymb,amsmath}
\usepackage{ifxetex,ifluatex}
\usepackage{fixltx2e} % provides \textsubscript
\ifnum 0\ifxetex 1\fi\ifluatex 1\fi=0 % if pdftex
  \usepackage[T1]{fontenc}
  \usepackage[utf8]{inputenc}
\else % if luatex or xelatex
  \ifxetex
    \usepackage{xltxtra,xunicode}
  \else
    \usepackage{fontspec}
  \fi
  \defaultfontfeatures{Mapping=tex-text,Scale=MatchLowercase}
  \newcommand{\euro}{€}
\fi
% use upquote if available, for straight quotes in verbatim environments
\IfFileExists{upquote.sty}{\usepackage{upquote}}{}
% use microtype if available
\IfFileExists{microtype.sty}{%
\usepackage{microtype}
\UseMicrotypeSet[protrusion]{basicmath} % disable protrusion for tt fonts
}{}
\ifxetex
  \usepackage[setpagesize=false, % page size defined by xetex
              unicode=false, % unicode breaks when used with xetex
              xetex]{hyperref}
\else
  \usepackage[unicode=true]{hyperref}
\fi
\usepackage[usenames,dvipsnames]{color}
\hypersetup{breaklinks=true,
            bookmarks=true,
            pdfauthor={Tian, Jiahe; Hu, Xiaoxiao; Huang, Jiani; Liu, Jiaxing; Shi, Ruixin; Wu, Chenning; Zhang, Cenyuan; Zhang, Yihan; Wang, Chen},
            pdftitle={ 黑盒测试设计与执行},
            colorlinks=true,
            citecolor=blue,
            urlcolor=blue,
            linkcolor=magenta,
            pdfborder={0 0 0}}
\urlstyle{same}  % don't use monospace font for urls
\setlength{\emergencystretch}{3em}  % prevent overfull lines
\providecommand{\tightlist}{%
  \setlength{\itemsep}{0pt}\setlength{\parskip}{0pt}}
\setcounter{secnumdepth}{5}

\title{\vspace{2in} 黑盒测试设计与执行\\\vspace{0.5em}{\large 软件质量保障与测试课程Lab5课程作业(第9组)}}
\author{Tian, Jiahe\footnote{Equal Contribution, Fudan University, 17307130313
  (\href{mailto:tianjh17@fudan.edu.cn}{\nolinkurl{tianjh17@fudan.edu.cn}})} \and Hu, Xiaoxiao\footnote{Equal Contribution, Fudan University, 17302010077
  (\href{mailto:xxhu17@fudan.edu.cn}{\nolinkurl{xxhu17@fudan.edu.cn}})} \and Huang, Jiani\footnote{Equal Contribution, Fudan University, 17302010063
  (\href{mailto:huangjn17@fudan.edu.cn}{\nolinkurl{huangjn17@fudan.edu.cn}})} \and Liu, Jiaxing\footnote{Equal Contribution, Fudan University, 17302010049
  (\href{mailto:jiaxingliu17@fudan.edu.cn}{\nolinkurl{jiaxingliu17@fudan.edu.cn}})} \and Shi, Ruixin\footnote{Equal Contribution, Fudan University, 17302010065
  (\href{mailto:rxshi17@fudan.edu.cn}{\nolinkurl{rxshi17@fudan.edu.cn}})} \and Wu, Chenning\footnote{Equal Contribution, Fudan University, 17302010066
  (\href{mailto:cnwu17@fudan.edu.cn}{\nolinkurl{cnwu17@fudan.edu.cn}})} \and Zhang, Cenyuan\footnote{Equal Contribution, Fudan University,
  17302010068
  (\href{mailto:cenyuanzhang17@fudan.edu.cn}{\nolinkurl{cenyuanzhang17@fudan.edu.cn}})} \and Zhang, Yihan\footnote{Equal Contribution, Fudan University, 17302010076
  (\href{mailto:zhangyihan17@fudan.edu.cn}{\nolinkurl{zhangyihan17@fudan.edu.cn}})} \and Wang, Chen\footnote{Equal Contribution, Fudan University, 16307110064
  (\href{mailto:wangc16@fudan.edu.cn}{\nolinkurl{wangc16@fudan.edu.cn}})}}
\date{2020年5月2日}



% Redefines (sub)paragraphs to behave more like sections
\ifx\paragraph\undefined\else
\let\oldparagraph\paragraph
\renewcommand{\paragraph}[1]{\oldparagraph{#1}\mbox{}}
\fi
\ifx\subparagraph\undefined\else
\let\oldsubparagraph\subparagraph
\renewcommand{\subparagraph}[1]{\oldsubparagraph{#1}\mbox{}}
\fi

\begin{document}
\maketitle

\newpage

\LARGE

\begin{center}
\textbf{黑盒测试设计与执行}
\end{center}

\large
\begin{center}
\textbf{\emph{软件质量保障与测试课程Lab5课程作业}}
\end{center}

\hypertarget{ux6458ux8981}{%
\section*{摘要}\label{ux6458ux8981}}
\addcontentsline{toc}{section}{摘要}

本次作业为软件质量保障与测试课程的Lab5课程作业,需要我们以小组为单位撰写在线出题考试系统的IEEE829测试文档.

\hypertarget{ux5173ux952eux8bcd}{%
\section*{关键词}\label{ux5173ux952eux8bcd}}
\addcontentsline{toc}{section}{关键词}

系统与软件工程; 系统与软件质量要求和评价; 测试文档

\normalsize

\newpage

\tableofcontents

\newpage

\hypertarget{ux6d4bux8bd5ux6846ux67b6ux8bbeux8ba1}{%
\section{测试框架设计}\label{ux6d4bux8bd5ux6846ux67b6ux8bbeux8ba1}}

\hypertarget{ux9ed1ux76d2ux6d4bux8bd5ux6846ux67b6ux56fe}{%
\subsection{黑盒测试框架图}\label{ux9ed1ux76d2ux6d4bux8bd5ux6846ux67b6ux56fe}}

\hypertarget{ux9ed1ux76d2ux6d4bux8bd5ux6846ux67b6ux56feux8bf4ux660e}{%
\subsection{黑盒测试框架图说明}\label{ux9ed1ux76d2ux6d4bux8bd5ux6846ux67b6ux56feux8bf4ux660e}}

\hypertarget{ux5173ux952eux5b57ux53caux6d4bux8bd5ux6570ux636eux8bbeux8ba1}{%
\section{关键字及测试数据设计}\label{ux5173ux952eux5b57ux53caux6d4bux8bd5ux6570ux636eux8bbeux8ba1}}

\hypertarget{ux6d4bux8bd5ux5173ux952eux5b57}{%
\subsection{测试关键字}\label{ux6d4bux8bd5ux5173ux952eux5b57}}

\hypertarget{ux767bux5f55}{%
\subsubsection{登录}\label{ux767bux5f55}}

\hypertarget{ux521bux5efaux8003ux9898}{%
\subsubsection{创建考题}\label{ux521bux5efaux8003ux9898}}

创建考题需要完成考题的相关设定。这些设定包括选择章节、选择知识点、设置作者、设置评审
、设置质管、设置类型、设置语言、设置出题日期、设置评审日期。而创建考题作为一个独立的UI测试,登录是一个基础环节。于是我们针对上述的关键步骤设计了以下关键字:
+ loginActions 登录系统 + navigateToAddQuestion 页面导航 +
showEditQuestion 选择添加考题 + chooseChapter 选择章节 +
chooseKnowledgePoint 选择知识点 + chooseAuthor 设置作者 + chooseReviewer
设置评审 + chooseQA 设置质管 + chooseType 设置题目类型 + startDate
设置出题开始日期 + finishDate 设置出题结束日期 + reviewStartDate
设置评审开始日期 + reviewFinishDate 设置评审结束日期 + chooseLanguage
选择语言 + saveQuestion 保存考题

其中我们没有将登录拆分为更加细致的行为,这是因为登录对于创建考题环节而言是比较基础和简单的。它不是本黑盒测试的考察内容。

\hypertarget{ux6d4bux8bd5ux6570ux636e}{%
\subsection{测试数据}\label{ux6d4bux8bd5ux6570ux636e}}

\hypertarget{ux767bux5f55-1}{%
\subsubsection{登录}\label{ux767bux5f55-1}}

\#\#\#\#等价类划分与边界值分析

\#\#\#\#测试数据

\hypertarget{ux521bux5efaux8003ux9898-1}{%
\subsubsection{创建考题}\label{ux521bux5efaux8003ux9898-1}}

\#\#\#\#等价类划分与边界值分析
创建考题的UI测试具备多个输入参数,并且这些参数的取值也多种多样。我们对输入参数进行等价类划分和边界值分析。

\textbf{等价类划分} \textbar{} 输入参数 \textbar{} 有效等价类 \textbar{}
无效等价类 \textbar{} \textbar{} ------------------ \textbar{}
----------------------------- \textbar{} ---------------------------
\textbar{} \textbar{} chapter \textbar{} {[}1{]}任何章节 \textbar{}
{[}2{]}为空 \textbar{} \textbar{} Knowledge point \textbar{}
{[}3{]}章节下任意知识点 \textbar{} {[}4{]}为空 \textbar{} \textbar{}
Author \textbar{} {[}5{]}任意用户 \textbar{} {[}6{]}为空 \textbar{}
\textbar{} Reviewer \textbar{} {[}7{]}除author外用户 \textbar{}
{[}8{]}author {[}9{]}为空 \textbar{} \textbar{} QA \textbar{}
{[}10{]}除author,reviewer外用户 \textbar{} {[}11{]}author或reviewer
{[}12{]}为空 \textbar{} \textbar{} Type \textbar{} {[}13{]}任意类型
\textbar{} {[}14{]}为空 \textbar{} \textbar{} Start date \textbar{}
{[}15{]}任意日期 \textbar{} {[}16{]}为空 \textbar{} \textbar{} Finish
date \textbar{} {[}17{]}在开始日期及之后的日期 \textbar{} {[}18{]}为空
\textbar{} \textbar{} Review start date \textbar{}
{[}19{]}出题开始日期及之后日期 \textbar{} {[}20{]}为空
{[}21{]}出题开始日期之前 \textbar{} \textbar{} Review finish date
\textbar{} {[}22{]}在评审开始日期及之后的日期,并且不早于出题结束日期
\textbar{} {[}23{]}为空 {[}24{]}出题结束日期之前 \textbar{} \textbar{}
language \textbar{} {[}25{]}任意语言 \textbar{} {[}26{]}为空 \textbar{}

边界值分析适用于具有连续取值的参数分析,题目中具有连续取值的只有出题日期与评审日期。
其中出题开始日期不存在边界限定,故不考虑。而对于评审结束日期,它的边界值与评审开始日期和出题结束日期相关。而这两个日期不存在约束关系,
故它的多个边界值条件可以同时成立。

\textbf{边界值分析} \textbar{} 输入参数 \textbar{} 边界值 \textbar{}
\textbar{} ------------------ \textbar{}
--------------------------------------------------------------------------------------------------------------
\textbar{} \textbar{} Finish date \textbar{} {[}27{]}与Start date相同
{[}28{]}Start date后一天 \textbar{} \textbar{} Review start date
\textbar{} {[}29{]}Start date前一天 {[}30{]}与Start date相同
{[}31{]}Start date后一天 \textbar{} \textbar{} Review finish date
\textbar{} {[}32{]}与Review start date相同 {[}33{]}Review start
date后一天 {[}34{]}Finish date前一天 {[}35{]}与Finish date相同
{[}36{]}Finish date后一天 \textbar{}

\#\#\#\#测试数据

\textbf{等价类划分} \textbar{} ID \textbar{} 覆盖的类 \textbar{} chapter
\textbar{} knowledge point \textbar{} author \textbar{} reviewer
\textbar{} QA \textbar{} type \textbar{} start date \textbar{} finish
date \textbar{} review start date \textbar{} review finish date
\textbar{} language \textbar{} 预期输出 \textbar{} \textbar{} ---
\textbar{} ---------------------------- \textbar{} ------- \textbar{}
--------------- \textbar{} --------- \textbar{} --------- \textbar{}
--------- \textbar{} ---- \textbar{} ---------- \textbar{} -----------
\textbar{} ----------------- \textbar{} ------------------ \textbar{}
-------- \textbar{} ---- \textbar{} \textbar{} 1 \textbar{}
1,3,5,7,10,13,15,17,19,22,25 \textbar{} 1 \textbar{} 1.2.1 \textbar{}
testadmin \textbar{} jmeter022 \textbar{} jmeter023 \textbar{} 情景题
\textbar{} 2020-05-03 \textbar{} 2020-05-03 \textbar{} 2020-05-03
\textbar{} 2020-05-03 \textbar{} 中文 \textbar{} 成功 \textbar{}
\textbar{} 2 \textbar{} 1,3,5,7,10,13,15,17,19,22,25 \textbar{} 1
\textbar{} 1.2.1 \textbar{} jmeter002 \textbar{} jmeter003 \textbar{}
jmeter004 \textbar{} 视频题 \textbar{} 2020-05-03 \textbar{} 2020-05-04
\textbar{} 2020-05-03 \textbar{} 2020-05-04 \textbar{} 英文 \textbar{}
成功 \textbar{} \textbar{} 3 \textbar{} 2 \textbar{} 空 \textbar{} 空
\textbar{} jmeter002 \textbar{} jmeter003 \textbar{} jmeter004
\textbar{} 视频题 \textbar{} 2020-05-03 \textbar{} 2020-05-04 \textbar{}
2020-05-03 \textbar{} 2020-05-04 \textbar{} 英文 \textbar{} 失败
\textbar{} \textbar{} 4 \textbar{} 4 \textbar{} 1 \textbar{} 空
\textbar{} jmeter002 \textbar{} jmeter003 \textbar{} jmeter004
\textbar{} 视频题 \textbar{} 2020-05-03 \textbar{} 2020-05-04 \textbar{}
2020-05-03 \textbar{} 2020-05-04 \textbar{} 英文 \textbar{} 失败
\textbar{} \textbar{} 5 \textbar{} 6 \textbar{} 1 \textbar{} 1.2.1
\textbar{} 空 \textbar{} jmeter022 \textbar{} jmeter023 \textbar{}
情景题 \textbar{} 2020-04-29 \textbar{} 2020-04-30 \textbar{} 2020-04-29
\textbar{} 2020-04-30 \textbar{} 中文 \textbar{} 失败 \textbar{}
\textbar{} 6 \textbar{} 9 \textbar{} 1 \textbar{} 1.2.1 \textbar{}
testadmin \textbar{} 空 \textbar{} jmeter023 \textbar{} 情景题
\textbar{} 2020-05-03 \textbar{} 2020-05-04 \textbar{} 2020-05-03
\textbar{} 2020-05-04 \textbar{} 中文 \textbar{} 失败 \textbar{}
\textbar{} 7 \textbar{} 8 \textbar{} 1 \textbar{} 1.2.1 \textbar{}
testadmin \textbar{} testadmin \textbar{} jmeter023 \textbar{} 情景题
\textbar{} 2020-05-03 \textbar{} 2020-05-04 \textbar{} 2020-05-03
\textbar{} 2020-05-04 \textbar{} 中文 \textbar{} 失败 \textbar{}
\textbar{} 8 \textbar{} 12 \textbar{} 1 \textbar{} 1.2.1 \textbar{}
testadmin \textbar{} jmeter022 \textbar{} 空 \textbar{} 情景题
\textbar{} 2020-05-03 \textbar{} 2020-05-04 \textbar{} 2020-05-03
\textbar{} 2020-05-04 \textbar{} 中文 \textbar{} 失败 \textbar{}
\textbar{} 9 \textbar{} 11 \textbar{} 1 \textbar{} 1.2.1 \textbar{}
testadmin \textbar{} jmeter022 \textbar{} jmeter022 \textbar{} 情景题
\textbar{} 2020-05-03 \textbar{} 2020-05-04 \textbar{} 2020-05-03
\textbar{} 2020-05-04 \textbar{} 中文 \textbar{} 失败 \textbar{}
\textbar{} 10 \textbar{} 14 \textbar{} 1 \textbar{} 1.2.1 \textbar{}
testadmin \textbar{} jmeter022 \textbar{} jmeter023 \textbar{} 空
\textbar{} 2020-05-03 \textbar{} 2020-05-04 \textbar{} 2020-05-03
\textbar{} 2020-05-04 \textbar{} 中文 \textbar{} 失败 \textbar{}
\textbar{} 11 \textbar{} 16 \textbar{} 1 \textbar{} 1.2.1 \textbar{}
testadmin \textbar{} jmeter022 \textbar{} jmeter023 \textbar{} 情景题
\textbar{} 空 \textbar{} 2020-05-04 \textbar{} 2020-05-03 \textbar{}
2020-05-04 \textbar{} 中文 \textbar{} 失败 \textbar{} \textbar{} 12
\textbar{} 18 \textbar{} 1 \textbar{} 1.2.1 \textbar{} testadmin
\textbar{} jmeter022 \textbar{} jmeter023 \textbar{} 情景题 \textbar{}
2020-05-03 \textbar{} 空 \textbar{} 2020-05-03 \textbar{} 2020-05-04
\textbar{} 中文 \textbar{} 失败 \textbar{} \textbar{} 13 \textbar{} 20
\textbar{} 1 \textbar{} 1.2.1 \textbar{} testadmin \textbar{} jmeter022
\textbar{} jmeter023 \textbar{} 情景题 \textbar{} 2020-05-03 \textbar{}
2020-05-04 \textbar{} 空 \textbar{} 2020-05-04 \textbar{} 中文
\textbar{} 失败 \textbar{} \textbar{} 14 \textbar{} 23 \textbar{} 1
\textbar{} 1.2.1 \textbar{} testadmin \textbar{} jmeter022 \textbar{}
jmeter023 \textbar{} 情景题 \textbar{} 2020-05-03 \textbar{} 2020-05-04
\textbar{} 2020-05-03 \textbar{} 空 \textbar{} 中文 \textbar{} 失败
\textbar{} \textbar{} 15 \textbar{} 26 \textbar{} 1 \textbar{} 1.2.1
\textbar{} testadmin \textbar{} jmeter022 \textbar{} jmeter023
\textbar{} 情景题 \textbar{} 2020-05-03 \textbar{} 2020-05-04 \textbar{}
2020-05-03 \textbar{} 2020-05-04 \textbar{} 空 \textbar{} 失败
\textbar{} \textbar{} 16 \textbar{} 21 \textbar{} 1 \textbar{} 1.2.1
\textbar{} testadmin \textbar{} jmeter022 \textbar{} jmeter023
\textbar{} 情景题 \textbar{} 2020-05-03 \textbar{} 2020-05-04 \textbar{}
2020-04-30 \textbar{} 2020-05-04 \textbar{} 中文 \textbar{} 失败
\textbar{} \textbar{} 17 \textbar{} 24 \textbar{} 1 \textbar{} 1.2.1
\textbar{} testadmin \textbar{} jmeter022 \textbar{} jmeter023
\textbar{} 情景题 \textbar{} 2020-05-03 \textbar{} 2020-05-04 \textbar{}
2020-05-03 \textbar{} 2020-05-03 \textbar{} 中文 \textbar{} 失败
\textbar{}

\textbf{边界值分析} \textbar{} ID \textbar{} 覆盖的类 \textbar{} chapter
\textbar{} knowledge point \textbar{} author \textbar{} reviewer
\textbar{} QA \textbar{} type \textbar{} start date \textbar{} finish
date \textbar{} review start date \textbar{} review finish date
\textbar{} language \textbar{} 预期输出 \textbar{} \textbar{} ---
\textbar{} -------------------- \textbar{} ------- \textbar{}
--------------- \textbar{} --------- \textbar{} --------- \textbar{}
--------- \textbar{} ---- \textbar{} ---------- \textbar{} -----------
\textbar{} ----------------- \textbar{} ------------------ \textbar{}
-------- \textbar{} ---- \textbar{} \textbar{} 18 \textbar{} 27,30,32,35
\textbar{} 1 \textbar{} 1.2.1 \textbar{} testadmin \textbar{} jmeter022
\textbar{} jmeter023 \textbar{} 情景题 \textbar{} 2020-05-03 \textbar{}
2020-05-03 \textbar{} 2020-05-03 \textbar{} 2020-05-03 \textbar{} 中文
\textbar{} 成功 \textbar{} \textbar{} 19 \textbar{} 28,31,33,36
\textbar{} 1 \textbar{} 1.2.1 \textbar{} testadmin \textbar{} jmeter022
\textbar{} jmeter023 \textbar{} 情景题 \textbar{} 2020-05-03 \textbar{}
2020-05-04 \textbar{} 2020-05-04 \textbar{} 2020-05-05 \textbar{} 中文
\textbar{} 成功 \textbar{} \textbar{} 20 \textbar{} 29 \textbar{} 1
\textbar{} 1.2.1 \textbar{} testadmin \textbar{} jmeter022 \textbar{}
jmeter023 \textbar{} 情景题 \textbar{} 2020-05-03 \textbar{} 2020-05-04
\textbar{} 2020-05-02 \textbar{} 2020-05-04 \textbar{} 中文 \textbar{}
失败 \textbar{} \textbar{} 21 \textbar{} 34 \textbar{} 1 \textbar{}
1.2.1 \textbar{} testadmin \textbar{} jmeter022 \textbar{} jmeter023
\textbar{} 情景题 \textbar{} 2020-05-03 \textbar{} 2020-05-04 \textbar{}
2020-05-03 \textbar{} 2020-05-03 \textbar{} 中文 \textbar{} 失败
\textbar{}

\hypertarget{ux6d4bux8bd5ux811aux672cux5b9eux73b0ux53caux8fd0ux884c}{%
\section{测试脚本实现及运行}\label{ux6d4bux8bd5ux811aux672cux5b9eux73b0ux53caux8fd0ux884c}}

\hypertarget{ux811aux672cux5b9eux73b0ux65b9ux5f0f}{%
\subsection{脚本实现方式}\label{ux811aux672cux5b9eux73b0ux65b9ux5f0f}}

\hypertarget{ux8fd0ux884cux622aux56feux53caux8bf4ux660e}{%
\subsection{运行截图及说明}\label{ux8fd0ux884cux622aux56feux53caux8bf4ux660e}}

\pagebreak

\hypertarget{ux53c2ux8003ux6587ux732e}{%
\section*{参考文献}\label{ux53c2ux8003ux6587ux732e}}
\addcontentsline{toc}{section}{参考文献}

\hypertarget{refs}{}
\leavevmode\hypertarget{ref-innovativeInternationalisation}{}%
International Organization for Standardization. 2014. \emph{Systems and
Software Engineering --- Systems and Software Quality Requirements and
Evaluation (SQuaRE) --- Guide to SQuaRE}. \emph{International
Organization for Standardization}. Vol. 2014.
\url{https://www.iso.org/standard/64764.html}.

\leavevmode\hypertarget{ref-innovative1}{}%
中国国家标准化管理委员会. 2016. \emph{GB/T
25000.51-2016《系统与软件工程系统与软件质量要求和评价 (SQuaRE) 第 51
部分 : 就绪可用软件产品 (RUSP) 的质量要求和测试细则》}.
\emph{系统与软件工程系统与软件质量要求和评价 (SQuaRE)}. Vol. 51.
中国国家标准化管理委员会. \url{http://openstd.samr.gov.cn}.

\leavevmode\hypertarget{ref-innovative3}{}%
---------. 2017a. \emph{GB/T 25000.12-2017《系统与软件工程
系统与软件质量要求和评价(SQuaRE) 第12部分:数据质量模型》}.
\emph{系统与软件工程系统与软件质量要求和评价 (SQuaRE)}. Vol. 12.
中国国家标准化管理委员会. \url{http://openstd.samr.gov.cn}.

\leavevmode\hypertarget{ref-innovative4}{}%
---------. 2017b. \emph{GB/T 25000.24-2017《系统与软件工程
系统与软件质量要求和评价(SQuaRE) 第24部分:数据质量测量》}.
\emph{系统与软件工程系统与软件质量要求和评价 (SQuaRE)}. Vol. 24.
中国国家标准化管理委员会. \url{http://openstd.samr.gov.cn}.

\leavevmode\hypertarget{ref-innovative5}{}%
---------. 2018. \emph{GB/T 25000.40-201《系统与软件工程
系统与软件质量要求和评价(SQuaRE) 第40部分:评价过程》}.
\emph{系统与软件工程系统与软件质量要求和评价 (SQuaRE)}. Vol. 40.
中国国家标准化管理委员会. \url{http://openstd.samr.gov.cn}.

\leavevmode\hypertarget{ref-innovative2}{}%
---------. 2019. \emph{GB/T 25000.23-2019《系统与软件工程
系统与软件质量要求和评价(SQuaRE) 第23部分:系统与软件产品质量测量》}.
\emph{系统与软件工程系统与软件质量要求和评价 (SQuaRE)}. Vol. 23.
中国国家标准化管理委员会. \url{http://openstd.samr.gov.cn}.

\end{document}
