\documentclass[hyperref, a4paper]{ctexart}
\usepackage{lmodern}
\usepackage{amssymb,amsmath}
\usepackage{ifxetex,ifluatex}
\usepackage{fixltx2e} % provides \textsubscript
\ifnum 0\ifxetex 1\fi\ifluatex 1\fi=0 % if pdftex
  \usepackage[T1]{fontenc}
  \usepackage[utf8]{inputenc}
\else % if luatex or xelatex
  \ifxetex
    \usepackage{xltxtra,xunicode}
  \else
    \usepackage{fontspec}
  \fi
  \defaultfontfeatures{Mapping=tex-text,Scale=MatchLowercase}
  \newcommand{\euro}{€}
\fi
% use upquote if available, for straight quotes in verbatim environments
\IfFileExists{upquote.sty}{\usepackage{upquote}}{}
% use microtype if available
\IfFileExists{microtype.sty}{%
\usepackage{microtype}
\UseMicrotypeSet[protrusion]{basicmath} % disable protrusion for tt fonts
}{}
\ifxetex
  \usepackage[setpagesize=false, % page size defined by xetex
              unicode=false, % unicode breaks when used with xetex
              xetex]{hyperref}
\else
  \usepackage[unicode=true]{hyperref}
\fi
\usepackage[usenames,dvipsnames]{color}
\hypersetup{breaklinks=true,
            bookmarks=true,
            pdfauthor={田嘉禾大佬},
            pdftitle={ 测试计划},
            colorlinks=true,
            citecolor=blue,
            urlcolor=blue,
            linkcolor=magenta,
            pdfborder={0 0 0}}
\urlstyle{same}  % don't use monospace font for urls
\setlength{\emergencystretch}{3em}  % prevent overfull lines
\providecommand{\tightlist}{%
  \setlength{\itemsep}{0pt}\setlength{\parskip}{0pt}}
\setcounter{secnumdepth}{5}

\title{\vspace{2in} 测试计划\\\vspace{0.5em}{\large A Report for Uniqlo Scholarship Winter Program (Supply Chain Reform
Direction)}}
\author{田嘉禾大佬\footnote{Undergraduate in Software Engineering, Software
  School of Fudan University; Software Development Engineer at Amazon
  Shanghai Institute of Artificial Intelligence, Amazon Web Services.
  (\href{mailto:wangc16@fudan.edu.cn}{\nolinkurl{wangc16@fudan.edu.cn}};
  \href{mailto:cwanam@amazon.com}{\nolinkurl{cwanam@amazon.com}})}}
\date{February 11th, 2020}



% Redefines (sub)paragraphs to behave more like sections
\ifx\paragraph\undefined\else
\let\oldparagraph\paragraph
\renewcommand{\paragraph}[1]{\oldparagraph{#1}\mbox{}}
\fi
\ifx\subparagraph\undefined\else
\let\oldsubparagraph\subparagraph
\renewcommand{\subparagraph}[1]{\oldsubparagraph{#1}\mbox{}}
\fi

\begin{document}
\maketitle

\newpage

\LARGE

\begin{center}
\textbf{A Comprehensive Research of the Supply Chain of Uniqlo}
\end{center}

\large
\begin{center}
\textbf{\emph{A Report for Uniqlo Scholarship Winter Program (Supply Chain Reform Direction)}}
\end{center}

\hypertarget{abstract}{%
\section*{Abstract}\label{abstract}}
\addcontentsline{toc}{section}{Abstract}

With the improvement of living standards, the demand of consumers for
new clothing has become ``fast fashion''. Both clothing varieties and
new styles require apparel companies to respond quickly to market
demands. However, Chinese companies' apparel industry supply chains are
facing slow problems. Since 2002, many international fast fashion brands
have begun to enter the Chinese apparel market, and these fast fashion
apparel brands have taken a unique marketing strategy and brought
different experiences to their customers, which is putting our apparel
brand at a disadvantage. As an excellent fast fashion apparel brand,
Uniqlo, which is the largest fast fashion brand in China, has made great
progress in China, achieving considerable economic benefits and relying
on unique marketing strategies.

One of the major advantages that differentiate Uniqlo from other apparel
companies is its proficient supply chain management. and the quick
response strategy is also an important supply chain management strategy.
In this essay, after a comprehensive analysis of the Uniqlo company, we
come to the conclusion that an excellent supply chain can bring vitality
to the company and make the entire company dynamic. I have also given
some personal insights about what other companies can learn from the
strategies which are adopted by Uniqlo and how these strategies can be
enhanced to have a better result.

\hypertarget{keywords}{%
\section*{Keywords}\label{keywords}}
\addcontentsline{toc}{section}{Keywords}

Supply Chain Management; Quick Response; Uniqlo

\normalsize

\newpage

\tableofcontents

\newpage

\hypertarget{ux529fux80fdux6027ux6d4bux8bd5}{%
\section{功能性测试}\label{ux529fux80fdux6027ux6d4bux8bd5}}

\hypertarget{ux767bux5f55ux53caux4e2aux4ebaux4fe1ux606fux6d4bux8bd5}{%
\subsection{登录及个人信息测试}\label{ux767bux5f55ux53caux4e2aux4ebaux4fe1ux606fux6d4bux8bd5}}

\begin{itemize}
\tightlist
\item
  用户输入正确用户名,密码登录测试
\item
  用户输入错误用户名,密码登录测试
\item
  修改个人信息测试
\end{itemize}

\hypertarget{ux6b65ux9aa4ux4e00ux5efaux7acbux4e00ux4e2aux65b0ux9879ux76ee}{%
\subsection{步骤一:建立一个新项目}\label{ux6b65ux9aa4ux4e00ux5efaux7acbux4e00ux4e2aux65b0ux9879ux76ee}}

\begin{itemize}
\tightlist
\item
  对创建新的出题系统、确定主持人进行测试
\item
  对为项目定义唯一项目名这一功能进行测试
\item
  对主持人进行的项目规划和运行进行测试
\item
  对创建新考题进行测试
\item
  对``开始''状态设置以及编写考题框架进行测试
\item
  对确定出题相关属性进行测试
\item
  对作者编写考题属性进行测试
\item
  对主持人设置考题状态``开始''权限进行测试
\item
  为考题添加单个作者,评审员,质管员进行测试
\item
  为考题添加多个作者,评审员,质管员进行测试
\item
  对作者和评审员之间读取互相信息权限进行测试
\item
  对作者和评审员的唯一标识符进行测试
\end{itemize}

\hypertarget{ux6b65ux9aa4ux4e8cux5f00ux59cbux542fux52a8ux9879ux76eeux72b6ux6001ux5f00ux59cbux89d2ux8272ux4e3bux6301ux4eba}{%
\subsection{步骤二:开始启动项目(状态:开始;角色:主持人)}\label{ux6b65ux9aa4ux4e8cux5f00ux59cbux542fux52a8ux9879ux76eeux72b6ux6001ux5f00ux59cbux89d2ux8272ux4e3bux6301ux4eba}}

\begin{itemize}
\item
  对主持人能够在状态为``开始''的考题内分配知识点、作者和评审员进行测试
\item
  对主持人能够为状态为``开始''的考题设定编写考题与评审考题的时间限期进行测试
\item
  对系统会对超出限期的任务报警进行测试
\item
  对主持人能对超出限期的任务提出警告或调整时间限期进行测试
\item
  对主持人能将分配了知识点、作者、评审员和限期的题目状态改为``编写'',状态改变后系统能给对应作者发送邮件的测试
\item
  对未完成分配任务的题目状态不能改为``编写''的测试
\item
  //对主持人能够批量将完成分配的题目状态改为``编写'',系统会给对应作者发送邮件的测试
\item
  对除主持人外,系统管理员、作者、评审员没有修改题目状态的权限的测试
\end{itemize}

\hypertarget{ux6b65ux9aa4ux4e09ux7f16ux5199ux8003ux9898ux72b6ux6001ux7f16ux5199ux89d2ux8272ux4f5cux8005}{%
\subsection{步骤三:编写考题(状态:编写;角色:作者)}\label{ux6b65ux9aa4ux4e09ux7f16ux5199ux8003ux9898ux72b6ux6001ux7f16ux5199ux89d2ux8272ux4f5cux8005}}

\begin{itemize}
\tightlist
\item
  对作者能够看到应当由他编写的考题的相关信息进行测试
\item
  对作者能够编写普通考题、给出标准答案并设置相关属性进行测试
\item
  对作者能够编写带有表格的题目、给出标准答案并设置相关属性进行测试
\item
  对作者能够编写带有图形的题目、给出标准答案并设置相关属性进行测试
\item
  对未完成编写(题干为空?),未给出标准答案或未完成相关属性设定的题目,作者不能修改其状态为``评审''的测试
\item
  对作者能够将编写完成、给出标准答案并设置相关属性的题目状态改为``评审'',系统会给评审员发出邮件的测试
\item
  对编写过程中主持人不能修改和操作的权限进行测试
\item
  对普通考题能以XML,Excel的形式导入出题系统的测试
\item
  对带有表格的题目能以XML,Excel的形式导入出题系统的测试
\item
  对带有图形的题目能以XML,Excel的形式导入出题系统的测试
\end{itemize}

\hypertarget{ux6b65ux9aa4ux56dbux8bc4ux5ba1ux8003ux9898ux72b6ux6001ux8bc4ux5ba1ux89d2ux8272ux8bc4ux5ba1ux5458}{%
\subsection{步骤四:评审考题(状态:评审;角色:评审员)}\label{ux6b65ux9aa4ux56dbux8bc4ux5ba1ux8003ux9898ux72b6ux6001ux8bc4ux5ba1ux89d2ux8272ux8bc4ux5ba1ux5458}}

\begin{itemize}
\tightlist
\item
  对评审员能够查看需要自己评审的题目的权限进行测试
\item
  对评审员能够查看需要再次评审的题目的权限进行测试
\item
  对评审员设置为``可接受''评审结果与``再审''状态的考题的流程进行测试
\item
  对评审员设置为``需修改''评审结果与``修改''状态的考题的流程进行测试
\item
  对评审员设置为``被拒绝''评审结果与``再审''状态的考题的流程进行测试
\item
  对评审过程中主持人没有修改和操作的权限进行测试
\end{itemize}

\hypertarget{ux6b65ux9aa4ux4e94ux518dux5ba1ux8003ux9898ux72b6ux6001ux518dux5ba1ux89d2ux8272ux8d28ux7ba1ux5458}{%
\subsection{步骤五:再审考题(状态:再审;角色:质管员)}\label{ux6b65ux9aa4ux4e94ux518dux5ba1ux8003ux9898ux72b6ux6001ux518dux5ba1ux89d2ux8272ux8d28ux7ba1ux5458}}

\begin{itemize}
\tightlist
\item
  对质管员将``可接受''评审结果的题目设置为``发布''状态的流程进行测试
\item
  对质管员将``被拒绝''评审结果的题目设置为``作废''状态的流程进行测试
\item
  对质管员将``可接受''评审结果的题目设置为``修改''状态的流程进行测试
\item
  对质管员阅读所有处于``再审''状态的考题的权限进行测试
\item
  对质管员给所有处于``再审''状态的考题编写评审意见和建议的权限进行测试
\item
  对质管员给所有处于``再审''状态的考题进行变更状态的权限进行测试
\item
  对再审过程中主持人没有修改和操作的权限进行测试
\end{itemize}

\hypertarget{ux6b65ux9aa4ux516dux4feeux6539ux8003ux9898ux72b6ux6001ux4feeux6539ux89d2ux8272ux4f5cux8005}{%
\subsection{步骤六:修改考题(状态:修改;角色:作者)}\label{ux6b65ux9aa4ux516dux4feeux6539ux8003ux9898ux72b6ux6001ux4feeux6539ux89d2ux8272ux4f5cux8005}}

\begin{itemize}
\tightlist
\item
  测试主持人不能进行修改题目
\item
  测试作者,评审员进行修改题目
\end{itemize}

\hypertarget{ux6b65ux9aa4ux4e03ux53d1ux5e03ux8003ux9898ux72b6ux6001ux53d1ux5e03ux89d2ux8272ux4e3bux6301ux4eba}{%
\subsection{步骤七:发布考题(状态:发布;角色:主持人)}\label{ux6b65ux9aa4ux4e03ux53d1ux5e03ux8003ux9898ux72b6ux6001ux53d1ux5e03ux89d2ux8272ux4e3bux6301ux4eba}}

\begin{itemize}
\tightlist
\item
  进行普通考题导出为XML,Excel测试
\item
  进行带表格或图片的考题导出为XML,Excel测试
\end{itemize}

\hypertarget{ux6b65ux9aa4ux516bux4f5cux5e9fux8003ux9898ux72b6ux6001ux4f5cux5e9fux89d2ux8272ux4e3bux6301ux4eba}{%
\subsection{步骤八:作废考题(状态:作废;角色:主持人)}\label{ux6b65ux9aa4ux516bux4f5cux5e9fux8003ux9898ux72b6ux6001ux4f5cux5e9fux89d2ux8272ux4e3bux6301ux4eba}}

\begin{itemize}
\tightlist
\item
  对主持人废除考题的操作进行测试
\item
  对主持人再次安排替补的考题这一流程进行测试
\item
  对主持人阅读处于作废状态的考题这一权限进行测试
\item
  对主持人没有修改和操作的权限进行测试
\end{itemize}

\hypertarget{ux6574ux4f53ux6d41ux7a0bux63a7ux5236}{%
\subsection{整体流程控制}\label{ux6574ux4f53ux6d41ux7a0bux63a7ux5236}}

// - 定义新的合法状态模型进行测试

// - 测试用户作出更改后系统所产生的记录信息可否追踪

\hypertarget{ux975eux529fux80fdux6027ux6d4bux8bd5}{%
\section{非功能性测试}\label{ux975eux529fux80fdux6027ux6d4bux8bd5}}

\hypertarget{ux6027ux80fdux6548ux7387}{%
\subsection{性能效率}\label{ux6027ux80fdux6548ux7387}}

\begin{itemize}
\tightlist
\item
  测试导入题库不同大小的数据耗时
\item
  测试从题库导出不同大小的数据耗时
\end{itemize}

\hypertarget{ux517cux5bb9ux6027}{%
\subsection{兼容性}\label{ux517cux5bb9ux6027}}

\begin{itemize}
\tightlist
\item
  针对不同浏览器做测试
\item
  针对不同用户端,不同窗口大小(如不同款式手机,平板电脑)的适配性做测试
\end{itemize}

\hypertarget{ux6613ux7528ux6027}{%
\subsection{易用性}\label{ux6613ux7528ux6027}}

\begin{itemize}
\tightlist
\item
  输入不符合规定格式的用户名/密码
\item
  对于合法值有限的下拉框,尝试抓包修改值为非法值之后发送
\item
  对于日期等格式固定的参数,尝试修改为不符合格式后发送
\item
  把不能为null的值修改为空后发送
\item
  测试搜索不存在的条目
\item
  测试不合法数据导入题库管理系统
\item
  测试上传图片的大小和类型限制
\item
  测试无效URL的图片
\item
  上传图片时测试上传非图片的文件
\item
  测试支持的图片文件大小
\item
  测试支持的表格行列极限大小
\item
  测试添加作者,评审,质管为不存在用户
\item
  定义新的不合法状态模型(如含有null字段,无意义权限等)进行测试
\end{itemize}

\hypertarget{ux53efux9760ux6027}{%
\subsection{可靠性}\label{ux53efux9760ux6027}}

\begin{itemize}
\tightlist
\item
  测试编辑到一半断电/掉线后的数据恢复能力
\item
  测试各角色对数据修改后未保存的后果
\item
  测试各角色对数据修改后能否撤销
\item
  测试用户作出更改后系统所产生的记录信息可否追踪
\item
  测试同一数据重复导入题库管理系统
\item
  测试未登录角色导入数据
\item
  测试未拥有权限的角色导入数据
\item
  在尚未有发布考题的时候进行导出测试
\item
  测试无权限的角色能否导出数据
\item
  测试未登录能否导出数据
\item
  测试多个用户在线且同时对同一考题进行更改对考题产生的作用
\item
  测试在用户更改某个考题时,另一用户删除该考题的后果
\item
  测试多位教师同时上传某一相同考题的后果
\item
  测试多位管理员同时导入同一数据的后果
\item
  测试多位管理员同时删除同一考题的后果
\item
  测试对网站进行DDoS攻击
\end{itemize}

\hypertarget{ux4fe1ux606fux5b89ux5168ux6027}{%
\subsection{信息安全性}\label{ux4fe1ux606fux5b89ux5168ux6027}}

\begin{itemize}
\tightlist
\item
  测试在各个环节通过SQL注入/XSS注入盗取数据/cookie
\item
  测试通过.DS\_Store,web.rar,.git等常见后缀是否能够泄露源码
\item
  测试网页前端注释中是否有敏感数据泄露
\item
  测试通过URL的越权访问
\item
  测试将含有XXE注入的XML数据导入题库管理系统
\item
  上传图片时测试上传嵌入webshell的文件
\item
  在登录界面用户名,密码或题目题干等处进行SQL注入,webshell插入,XSS注入
\end{itemize}

\hypertarget{ux7ef4ux62a4ux6027}{%
\subsection{维护性}\label{ux7ef4ux62a4ux6027}}

\hypertarget{ux53efux79fbux690dux6027}{%
\subsection{可移植性}\label{ux53efux79fbux690dux6027}}

\begin{itemize}
\tightlist
\item
  在Mac,Windows,Linux等各个平台上独立测试
\end{itemize}

\pagebreak

\hypertarget{bibliography}{%
\section*{Bibliography}\label{bibliography}}
\addcontentsline{toc}{section}{Bibliography}

\hypertarget{refs}{}
\leavevmode\hypertarget{ref-innovative4}{}%
Agarwal, Sanjeev, and Sridhar N. Ramaswami. 1992. ``Choice of Foreign
Market Entry Mode: Impact of Ownership, Location and Internalization
Factors.'' \emph{Journal of International Business Studies} 23 (1).
Palgrave Macmillan UK: 1--27.
\url{https://doi.org/10.1057/palgrave.jibs.8490257}.

\leavevmode\hypertarget{ref-innovative5}{}%
Andersen, Otto, and Low Suat Kheam. 1998. ``Resource-Based Theory and
International Growth Strategies: An Exploratory Study.''
\emph{International Business Review} 7 (2). Elsevier B.V.: 163--84.
\url{https://doi.org/10.1016/S0969-5931(98)00004-3}.

\leavevmode\hypertarget{ref-innovative1}{}%
Andersson, Svante. 2000. ``The Internationalization of the Firm from an
Entrepreneurial Perspective.'' \emph{International Studies of Management
\& Organization} 30 (1). Taylor \& Francis, Ltd.: 63--92.
\url{https://doi.org/10.1080/00208825.2000.11656783}.

\leavevmode\hypertarget{ref-innovativeInternationalisation}{}%
Andersson, Svante, and Ingemar Wictor. 2003. ``Innovative
Internationalisation in New Firms: Born Globals--the Swedish Case.''
\emph{Journal of International Entrepreneurship} 11 (1). Kluwer Academic
Publishers: 249--75. \url{https://doi.org/10.1023/A:1024110806241}.

\leavevmode\hypertarget{ref-innovative7}{}%
Childs, Michelle Lynn, and Byoungho Jin. 2014. ``Is Uppsala Model Valid
to Fashion Retailers? An Analysis from Internationalisation Patterns of
Fast Fashion Retailers.'' \emph{Journal of Fashion Marketing and
Management} 18 (16). Emerald Publishing Limited: 36--51.
\url{https://doi.org/10.1108/JFMM-10-2012-0061}.

\leavevmode\hypertarget{ref-innovative2}{}%
Cox, Andrew. 1999. ``Power, Value and Supply Chain Management.''
\emph{Supply Chain Management} 4 (4). MCB UP Ltd: 167--75.
\url{https://doi.org/10.1108/13598549910284480}.

\leavevmode\hypertarget{ref-innovative8}{}%
Dibb, Sally. 1996. ``The Impact of the Changing Marketing Environment in
the Pacific Rim: Four Case Studies.'' \emph{International Journal of
Retail \& Distribution Management} 24 (1). MCB University Press: 16--29.
\url{https://doi.org/10.1108/09590559610131691}.

\leavevmode\hypertarget{ref-innovative9}{}%
Dunning, John H. 1977. ``Trade, Location of Economic Activity and the
MNE: A Search for an Eclectic Approach.'' \emph{The International
Allocation of Economic Activity} 30 (1). Palgrave Macmillan, London:
395--418. \url{https://doi.org/10.1007/978-1-349-03196-2_38}.

\leavevmode\hypertarget{ref-innovative10}{}%
---------. 1988. ``The Eclectic Paradigm of International Production: A
Restatement and Some Possible Extensions.'' \emph{Journal of
International Business Studies} 19 (1). Palgrave Macmillan UK: 1--31.
\url{https://doi.org/10.1057/palgrave.jibs.8490372}.

\leavevmode\hypertarget{ref-innovative3}{}%
Edition Department of Hanbai Kakushin. 2000. \emph{ABC Kaikaku No Zenbou
(the Full Picture of All Better Change Activity)}. \emph{Hnbai
Kakushin}.

\leavevmode\hypertarget{ref-innovative11}{}%
Forsgren, Mats. 2015. ``The Concept of Learning in the Uppsala
Internationalization Process Model: A Critical Review.''
\emph{Knowledge, Networks and Power} 19 (1). Palgrave Macmillan, London:
88--110. \url{https://doi.org/10.1057/9781137508829_4}.

\leavevmode\hypertarget{ref-innovative12}{}%
Hayes, S.G., and Nicola Jones. 2006. ``Fast Fashion: A Financial
Snapshot.'' \emph{Journal of Fashion Marketing and Management} 10 (3).
Emerald Group Publishing Limited: 282--300.
\url{https://doi.org/10.1108/13612020610679277}.

\leavevmode\hypertarget{ref-innovative13}{}%
Jackson, Paul, and Leigh Sparks. 2005. ``Retail Internationalisation:
Marks and Spencer in Hong Kong.'' \emph{International Journal of Retail
\& Distribution Management} 33 (10). Emerald Group Publishing Limited:
766--83. \url{https://doi.org/10.1108/09590550510622308}.

\leavevmode\hypertarget{ref-innovative14}{}%
Johanson, Jan, and Jan-Erik Vahlne. 1977. ``The Internationalization
Process of the Firm --- A Model of Knowledge Development and Increasing
Foreign Market Commitments.'' \emph{Journal of International Business
Studies} 10 (3). Palgrave Macmillan, London: 23--32.
\url{https://doi.org/10.1057/palgrave.jibs.8490676}.

\leavevmode\hypertarget{ref-innovative16}{}%
Lopez, Carmen, and Ying Fan. 2009. ``Internationalisation of the Spanish
Fashion Brand Zara.'' \emph{Journal of Fashion Marketing and Management}
13 (2). Emerald Group Publishing Limited: 279--96.
\url{https://doi.org/10.1108/13612020910957770}.

\leavevmode\hypertarget{ref-innovative15}{}%
Saxena, Ravindra P., and Pradeep K. Khandelwal. 2010. ``Is the Magic of
`Feel Good' and `Look Great' at Giordano Still Working?''
\emph{Management Decision} 48 (3). Emerald Group Publishing Limited:
440--55. \url{https://doi.org/10.1108/00251741011037792}.

\leavevmode\hypertarget{ref-innovative6}{}%
Vertica, Bhardwaj, Megan, Eickman, and Rodney. 2011. ``A Case Study on
the Internationalization Process of a `Born-Global' Fashion Retailer.''
\emph{International Review of Retail, Distribution and Consumer
Research}, January. Taylor \& Francis (Routledge).

\end{document}
