\documentclass[hyperref, a4paper]{ctexart}
\usepackage{lmodern}
\usepackage{amssymb,amsmath}
\usepackage{ifxetex,ifluatex}
\usepackage{fixltx2e} % provides \textsubscript
\ifnum 0\ifxetex 1\fi\ifluatex 1\fi=0 % if pdftex
  \usepackage[T1]{fontenc}
  \usepackage[utf8]{inputenc}
\else % if luatex or xelatex
  \ifxetex
    \usepackage{xltxtra,xunicode}
  \else
    \usepackage{fontspec}
  \fi
  \defaultfontfeatures{Mapping=tex-text,Scale=MatchLowercase}
  \newcommand{\euro}{€}
\fi
% use upquote if available, for straight quotes in verbatim environments
\IfFileExists{upquote.sty}{\usepackage{upquote}}{}
% use microtype if available
\IfFileExists{microtype.sty}{%
\usepackage{microtype}
\UseMicrotypeSet[protrusion]{basicmath} % disable protrusion for tt fonts
}{}
\ifxetex
  \usepackage[setpagesize=false, % page size defined by xetex
              unicode=false, % unicode breaks when used with xetex
              xetex]{hyperref}
\else
  \usepackage[unicode=true]{hyperref}
\fi
\usepackage[usenames,dvipsnames]{color}
\hypersetup{breaklinks=true,
            bookmarks=true,
            pdfauthor={Tian, Jiahe; Hu, Xiaoxiao; Huang, Jiani; Liu, Jiaxing; Shi, Ruixin; Wu, Chenning; Zhang, Cenyuan; Zhang, Yihan; Wang, Chen},
            pdftitle={ 性能测试设计与执行},
            colorlinks=true,
            citecolor=blue,
            urlcolor=blue,
            linkcolor=magenta,
            pdfborder={0 0 0}}
\urlstyle{same}  % don't use monospace font for urls
\usepackage{graphicx,grffile}
\makeatletter
\def\maxwidth{\ifdim\Gin@nat@width>\linewidth\linewidth\else\Gin@nat@width\fi}
\def\maxheight{\ifdim\Gin@nat@height>\textheight\textheight\else\Gin@nat@height\fi}
\makeatother
% Scale images if necessary, so that they will not overflow the page
% margins by default, and it is still possible to overwrite the defaults
% using explicit options in \includegraphics[width, height, ...]{}
\setkeys{Gin}{width=\maxwidth,height=\maxheight,keepaspectratio}
\setlength{\emergencystretch}{3em}  % prevent overfull lines
\providecommand{\tightlist}{%
  \setlength{\itemsep}{0pt}\setlength{\parskip}{0pt}}
\setcounter{secnumdepth}{5}

\title{\vspace{2in} 性能测试设计与执行\\\vspace{0.5em}{\large 软件质量保障与测试课程Lab6课程作业(第9组)}}
\author{Tian, Jiahe\footnote{Equal Contribution, Fudan University, 17307130313
  (\href{mailto:tianjh17@fudan.edu.cn}{\nolinkurl{tianjh17@fudan.edu.cn}})} \and Hu, Xiaoxiao\footnote{Equal Contribution, Fudan University, 17302010077
  (\href{mailto:xxhu17@fudan.edu.cn}{\nolinkurl{xxhu17@fudan.edu.cn}})} \and Huang, Jiani\footnote{Equal Contribution, Fudan University, 17302010063
  (\href{mailto:huangjn17@fudan.edu.cn}{\nolinkurl{huangjn17@fudan.edu.cn}})} \and Liu, Jiaxing\footnote{Equal Contribution, Fudan University, 17302010049
  (\href{mailto:jiaxingliu17@fudan.edu.cn}{\nolinkurl{jiaxingliu17@fudan.edu.cn}})} \and Shi, Ruixin\footnote{Equal Contribution, Fudan University, 17302010065
  (\href{mailto:rxshi17@fudan.edu.cn}{\nolinkurl{rxshi17@fudan.edu.cn}})} \and Wu, Chenning\footnote{Equal Contribution, Fudan University, 17302010066
  (\href{mailto:cnwu17@fudan.edu.cn}{\nolinkurl{cnwu17@fudan.edu.cn}})} \and Zhang, Cenyuan\footnote{Equal Contribution, Fudan University,
  17302010068
  (\href{mailto:cenyuanzhang17@fudan.edu.cn}{\nolinkurl{cenyuanzhang17@fudan.edu.cn}})} \and Zhang, Yihan\footnote{Equal Contribution, Fudan University, 17302010076
  (\href{mailto:zhangyihan17@fudan.edu.cn}{\nolinkurl{zhangyihan17@fudan.edu.cn}})} \and Wang, Chen\footnote{Equal Contribution, Fudan University, 16307110064
  (\href{mailto:wangc16@fudan.edu.cn}{\nolinkurl{wangc16@fudan.edu.cn}})}}
\date{2020年5月14日}



% Redefines (sub)paragraphs to behave more like sections
\ifx\paragraph\undefined\else
\let\oldparagraph\paragraph
\renewcommand{\paragraph}[1]{\oldparagraph{#1}\mbox{}}
\fi
\ifx\subparagraph\undefined\else
\let\oldsubparagraph\subparagraph
\renewcommand{\subparagraph}[1]{\oldsubparagraph{#1}\mbox{}}
\fi

\begin{document}
\maketitle

\newpage

\LARGE

\begin{center}
\textbf{性能测试设计与执行}
\end{center}

\large
\begin{center}
\textbf{\emph{软件质量保障与测试课程Lab6课程作业}}
\end{center}

\hypertarget{ux6458ux8981}{%
\section*{摘要}\label{ux6458ux8981}}
\addcontentsline{toc}{section}{摘要}

本次作业为软件质量保障与测试课程的Lab6课程作业,需要我们以小组为单位完成对出题系统的性能测试。本文档分为两小节。第一小节介绍了本小组进行性能测试采用的策略;第二小节介绍了性能测试的结果及系统性能分析。

\hypertarget{ux5173ux952eux8bcd}{%
\section*{关键词}\label{ux5173ux952eux8bcd}}
\addcontentsline{toc}{section}{关键词}

系统与软件工程; 系统与软件质量要求和评价; 测试文档

\normalsize

\newpage

\tableofcontents

\newpage

\hypertarget{ux6d4bux8bd5ux7b56ux7565}{%
\section{测试策略}\label{ux6d4bux8bd5ux7b56ux7565}}

本次测试的对象是出题系统中的登录和创建考题功能。登录功能分为认证以及选择项目两部分,创建考题功能则包括主持人创建新考题,完成属性配置、角色分配等活动,到将考题保存为止。性能指标为系统同时在线100人,20个并发访问。

为了产生必要的负载,需要依赖工具来进行性能测试。鉴于作为测试对象的出题系统是一个web应用,所以选择了开源的测试工具Jmeter来进行测试。通过Jmeter产生的聚合报告中的响应时间、吞吐量等来对系统性能进行分析。

测试中采用的策略包括并发测试和负载测试。并发测试主要用于检验系统是否能够达到给定的性能指标,即在给定条件下相应用户输入的能力是否达到要求。负载测试中则是通过不同并发数下的测试,对比响应时间、吞吐量等的变化,来分析系统性能的可扩展性,系统处理能力何时达到饱和状态,以及观察在并发数超出给定指标的情况下,系统能否继续正常运行。

并发测试中,对应在线100人的要求的是发出100个登录请求;对应20个并发访问的要求,发出20个创建考题的请求。我们使用了Jmeter工具中的定时器来让请求同时发送,以观察并发状况下服务的行为表现。同一时间对后端服务进行调用能更好地发现资源竞争、资源死锁等问题。

负载测试中,则是调整了并发数,将不同并发数下的聚合报告。对于在线人数的要求,进行了从20到140个用户同时在线的测试,每相隔20数量进行一次测试。对于并发访问的要求,进行了创建20到100个考题的测试,每次增量为20。通过这一策略,逐步增加系统的负载,直到超出指标,来寻找系统的性能上限,系统的处理能力,及系统在高负载情况下的稳定性。

\hypertarget{ux7cfbux7edfux6027ux80fdux53caux6d4bux8bd5ux7ed3ux679cux5206ux6790}{%
\section{系统性能及测试结果分析}\label{ux7cfbux7edfux6027ux80fdux53caux6d4bux8bd5ux7ed3ux679cux5206ux6790}}

\hypertarget{ux767bux5f55}{%
\subsection{登录}\label{ux767bux5f55}}

\begin{enumerate}
\def\labelenumi{\arabic{enumi}.}
\tightlist
\item
  100 个用户同时在线
\end{enumerate}

\includegraphics{resources/wcn/login_100.png}

从结果中可以看出,在并发数为100的情况下,认证步骤的平均响应时间为9.4秒,选择项目的平均响应时间为1.1秒,整个登录步骤的平均响应时间约为10.5秒。根据用户满意度曲线来看,在并发数100的情况下,系统的响应时间有些过长,性能指标没有很好的达到。

\begin{enumerate}
\def\labelenumi{\arabic{enumi}.}
\setcounter{enumi}{1}
\tightlist
\item
  20 ~ 140 个用户同时在线,每次递增20个用户,对系统性能进行并发测试
\end{enumerate}

\includegraphics{resources/wcn/login_aggregate.png}

在上一小节中,我们小组根据助教给出的指标进行了测试,发现测试的情况并不能达到在100并发的情况下有较好的响应性能。因此,我们小组进一步进行了更为深入的并发测试。在这一测试中,我们选取了上一小节中测试的并发数的20\%作为初始值,并以上述并发数的20\%作为步长进行阶梯并发测试,最终以上述并发数目的140\%作为终止值。从而进一步探究这一系统在不同的并发数下的性能情况,并从这一测试的结果中得到这一系统能够支撑的较大并发数目。

下面对这一部分测试的结果情况进行分析:

从表中可以看出,登录的选择部分的响应时间随着并发数的增加变化不大,说明这一服务有较大的可扩展性。而认证部分随着并发数增加,响应时间明显变长。所以对于登录功能来说,认证部分是性能的瓶颈所在,调优时应重点关注。

登录部分的性能测试未能达到指标,可能是由于在本地服务器上进行的测试。在并发数达到80后,吞吐量不再有太大的增长,说明系统处理能力已经接近饱和,但并发数继续增加,直到达到指标并发数140\%的情况下,虽然响应时间变长,但系统仍然能保持功能完整性,没有失效,说明系统在压力下仍能正常运行,稳定性较好。

\hypertarget{ux521bux5efaux8003ux9898}{%
\subsection{创建考题}\label{ux521bux5efaux8003ux9898}}

\begin{enumerate}
\def\labelenumi{\arabic{enumi}.}
\tightlist
\item
  20道考题的并发创建
\end{enumerate}

\includegraphics{resources/wcn/create_question_20.png}

从结果中可以看出,在并发数为20的情况下,
响应时间平均为1.9秒,最大不超过2.5秒。所以创建考题部分较好地达到了性能指标。

\begin{enumerate}
\def\labelenumi{\arabic{enumi}.}
\setcounter{enumi}{1}
\tightlist
\item
  20 \textasciitilde{} 100
  道考题并发创建,每次递增20道考题创建量,对系统性能进行并发测试
\end{enumerate}

\includegraphics{resources/wcn/create_question_aggregate.png}

上一小节的测试中,根据性能指标中的要求,我们小组进行了并发数为20的测试。在这一测试中,系统有着较好的响应性能。在这一节中,基于负载测试的策略,我们以20为并发数的初始值,步长为20进行了阶梯并发测试,来对这一服务的性能进行进一步的了解。从这一部分测试结果可以看出,随着并发数的上升,响应时间有明显的增加。但在超过指标100\%以上的情况下,服务仍然能够正常运行,且以用户满意度曲线为参考,响应时间较为合理,所以这一服务拥有很高的可扩展性。

\hypertarget{ux670dux52a1ux5668ux6027ux80fd}{%
\subsection{服务器性能}\label{ux670dux52a1ux5668ux6027ux80fd}}

本次性能测试服务器使用本地服务器,运行环境如下:

\includegraphics{resources/wcn/environment.png}

\pagebreak

\hypertarget{ux53c2ux8003ux6587ux732e}{%
\section*{参考文献}\label{ux53c2ux8003ux6587ux732e}}
\addcontentsline{toc}{section}{参考文献}

\hypertarget{refs}{}
\leavevmode\hypertarget{ref-innovativeInternationalisation}{}%
International Organization for Standardization. 2014. \emph{Systems and
Software Engineering --- Systems and Software Quality Requirements and
Evaluation (SQuaRE) --- Guide to SQuaRE}. \emph{International
Organization for Standardization}. Vol. 2014.
\url{https://www.iso.org/standard/64764.html}.

\leavevmode\hypertarget{ref-innovative1}{}%
中国国家标准化管理委员会. 2016. \emph{GB/T
25000.51-2016《系统与软件工程系统与软件质量要求和评价 (SQuaRE) 第 51
部分 : 就绪可用软件产品 (RUSP) 的质量要求和测试细则》}.
\emph{系统与软件工程系统与软件质量要求和评价 (SQuaRE)}. Vol. 51.
中国国家标准化管理委员会. \url{http://openstd.samr.gov.cn}.

\leavevmode\hypertarget{ref-innovative3}{}%
---------. 2017a. \emph{GB/T 25000.12-2017《系统与软件工程
系统与软件质量要求和评价(SQuaRE) 第12部分:数据质量模型》}.
\emph{系统与软件工程系统与软件质量要求和评价 (SQuaRE)}. Vol. 12.
中国国家标准化管理委员会. \url{http://openstd.samr.gov.cn}.

\leavevmode\hypertarget{ref-innovative4}{}%
---------. 2017b. \emph{GB/T 25000.24-2017《系统与软件工程
系统与软件质量要求和评价(SQuaRE) 第24部分:数据质量测量》}.
\emph{系统与软件工程系统与软件质量要求和评价 (SQuaRE)}. Vol. 24.
中国国家标准化管理委员会. \url{http://openstd.samr.gov.cn}.

\leavevmode\hypertarget{ref-innovative5}{}%
---------. 2018. \emph{GB/T 25000.40-201《系统与软件工程
系统与软件质量要求和评价(SQuaRE) 第40部分:评价过程》}.
\emph{系统与软件工程系统与软件质量要求和评价 (SQuaRE)}. Vol. 40.
中国国家标准化管理委员会. \url{http://openstd.samr.gov.cn}.

\leavevmode\hypertarget{ref-innovative2}{}%
---------. 2019. \emph{GB/T 25000.23-2019《系统与软件工程
系统与软件质量要求和评价(SQuaRE) 第23部分:系统与软件产品质量测量》}.
\emph{系统与软件工程系统与软件质量要求和评价 (SQuaRE)}. Vol. 23.
中国国家标准化管理委员会. \url{http://openstd.samr.gov.cn}.

\end{document}
