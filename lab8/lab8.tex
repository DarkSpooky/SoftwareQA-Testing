\documentclass[hyperref, a4paper]{ctexart}
\usepackage{lmodern}
\usepackage{amssymb,amsmath}
\usepackage{ifxetex,ifluatex}
\usepackage{fixltx2e} % provides \textsubscript
\ifnum 0\ifxetex 1\fi\ifluatex 1\fi=0 % if pdftex
  \usepackage[T1]{fontenc}
  \usepackage[utf8]{inputenc}
\else % if luatex or xelatex
  \ifxetex
    \usepackage{xltxtra,xunicode}
  \else
    \usepackage{fontspec}
  \fi
  \defaultfontfeatures{Mapping=tex-text,Scale=MatchLowercase}
  \newcommand{\euro}{€}
\fi
% use upquote if available, for straight quotes in verbatim environments
\IfFileExists{upquote.sty}{\usepackage{upquote}}{}
% use microtype if available
\IfFileExists{microtype.sty}{%
\usepackage{microtype}
\UseMicrotypeSet[protrusion]{basicmath} % disable protrusion for tt fonts
}{}
\ifxetex
  \usepackage[setpagesize=false, % page size defined by xetex
              unicode=false, % unicode breaks when used with xetex
              xetex]{hyperref}
\else
  \usepackage[unicode=true]{hyperref}
\fi
\usepackage[usenames,dvipsnames]{color}
\hypersetup{breaklinks=true,
            bookmarks=true,
            pdfauthor={Tian, Jiahe; Hu, Xiaoxiao; Huang, Jiani; Liu, Jiaxing; Shi, Ruixin; Wu, Chenning; Zhang, Cenyuan; Zhang, Yihan; Wang, Chen},
            pdftitle={ 场景测试报告},
            colorlinks=true,
            citecolor=blue,
            urlcolor=blue,
            linkcolor=magenta,
            pdfborder={0 0 0}}
\urlstyle{same}  % don't use monospace font for urls
\setlength{\emergencystretch}{3em}  % prevent overfull lines
\providecommand{\tightlist}{%
  \setlength{\itemsep}{0pt}\setlength{\parskip}{0pt}}
\setcounter{secnumdepth}{5}

\title{\vspace{2in} 场景测试报告\\\vspace{0.5em}{\large 软件质量保障与测试课程Lab8课程作业(第9组)}}
\author{Tian, Jiahe\footnote{Equal Contribution, Fudan University, 17307130313
  (\href{mailto:tianjh17@fudan.edu.cn}{\nolinkurl{tianjh17@fudan.edu.cn}})} \and Hu, Xiaoxiao\footnote{Equal Contribution, Fudan University, 17302010077
  (\href{mailto:xxhu17@fudan.edu.cn}{\nolinkurl{xxhu17@fudan.edu.cn}})} \and Huang, Jiani\footnote{Equal Contribution, Fudan University, 17302010063
  (\href{mailto:huangjn17@fudan.edu.cn}{\nolinkurl{huangjn17@fudan.edu.cn}})} \and Liu, Jiaxing\footnote{Equal Contribution, Fudan University, 17302010049
  (\href{mailto:jiaxingliu17@fudan.edu.cn}{\nolinkurl{jiaxingliu17@fudan.edu.cn}})} \and Shi, Ruixin\footnote{Equal Contribution, Fudan University, 17302010065
  (\href{mailto:rxshi17@fudan.edu.cn}{\nolinkurl{rxshi17@fudan.edu.cn}})} \and Wu, Chenning\footnote{Equal Contribution, Fudan University, 17302010066
  (\href{mailto:cnwu17@fudan.edu.cn}{\nolinkurl{cnwu17@fudan.edu.cn}})} \and Zhang, Cenyuan\footnote{Equal Contribution, Fudan University,
  17302010068
  (\href{mailto:cenyuanzhang17@fudan.edu.cn}{\nolinkurl{cenyuanzhang17@fudan.edu.cn}})} \and Zhang, Yihan\footnote{Equal Contribution, Fudan University, 17302010076
  (\href{mailto:zhangyihan17@fudan.edu.cn}{\nolinkurl{zhangyihan17@fudan.edu.cn}})} \and Wang, Chen\footnote{Equal Contribution, Fudan University, 16307110064
  (\href{mailto:wangc16@fudan.edu.cn}{\nolinkurl{wangc16@fudan.edu.cn}})}}
\date{2020年5月31日}



% Redefines (sub)paragraphs to behave more like sections
\ifx\paragraph\undefined\else
\let\oldparagraph\paragraph
\renewcommand{\paragraph}[1]{\oldparagraph{#1}\mbox{}}
\fi
\ifx\subparagraph\undefined\else
\let\oldsubparagraph\subparagraph
\renewcommand{\subparagraph}[1]{\oldsubparagraph{#1}\mbox{}}
\fi

\begin{document}
\maketitle

\newpage

\LARGE

\begin{center}
\textbf{场景测试报告}
\end{center}

\large
\begin{center}
\textbf{\emph{软件质量保障与测试课程Lab8课程作业}}
\end{center}

\hypertarget{ux6458ux8981}{%
\section*{摘要}\label{ux6458ux8981}}
\addcontentsline{toc}{section}{摘要}

本次作业为软件质量保障与测试课程的Lab8课程作业,需要我们以小组为单位完成对出题系统的场景测试。本文档分为五小节。第一小节介绍了本小组进行测试计划设计的情况;第二小节介绍了本小组设计测试范围的情况;第三小节整理了本小组进行测试的结果;第四小节介绍了本小组对bug等级的评估情况;第五小节介绍了本小组对本次场景测试的测试总结。

\hypertarget{ux5173ux952eux8bcd}{%
\section*{关键词}\label{ux5173ux952eux8bcd}}
\addcontentsline{toc}{section}{关键词}

系统与软件工程; 系统与软件质量要求和评价; 测试文档

\normalsize

\newpage

\tableofcontents

\newpage

\hypertarget{ux6d4bux8bd5ux8ba1ux5212}{%
\section{测试计划}\label{ux6d4bux8bd5ux8ba1ux5212}}

\hypertarget{ux767bux5f55ux53caux4e2aux4ebaux4fe1ux606f}{%
\subsection{登录及个人信息}\label{ux767bux5f55ux53caux4e2aux4ebaux4fe1ux606f}}

\hypertarget{ux767bux5f55}{%
\subsubsection{登录}\label{ux767bux5f55}}

\begin{itemize}
\tightlist
\item
  输入正确的用户名和密码
\item
  输入错误的用户名和密码
\end{itemize}

\hypertarget{ux4feeux6539ux4e2aux4ebaux4fe1ux606f}{%
\subsubsection{修改个人信息}\label{ux4feeux6539ux4e2aux4ebaux4fe1ux606f}}

\begin{itemize}
\tightlist
\item
  登录状态下对用户名进行修改
\item
  登录状态下对密码进行修改
\item
  非登录状态下访问修改个人信息页面
\end{itemize}

\hypertarget{ux5efaux7acbux65b0ux9879ux76ee}{%
\subsection{建立新项目}\label{ux5efaux7acbux65b0ux9879ux76ee}}

\hypertarget{ux521bux5efaux9879ux76ee}{%
\subsubsection{创建项目}\label{ux521bux5efaux9879ux76ee}}

\begin{itemize}
\tightlist
\item
  主持人登录后创建项目,确定项目名称,项目规划
\end{itemize}

\hypertarget{ux65b0ux5efaux8003ux9898}{%
\subsection{新建考题}\label{ux65b0ux5efaux8003ux9898}}

\hypertarget{ux521bux5efaux8003ux9898}{%
\subsubsection{创建考题}\label{ux521bux5efaux8003ux9898}}

\begin{itemize}
\tightlist
\item
  主持人通过创建按钮创建新考题
\end{itemize}

\hypertarget{ux8003ux9898ux8bbeux7f6e}{%
\subsubsection{考题设置}\label{ux8003ux9898ux8bbeux7f6e}}

\begin{itemize}
\tightlist
\item
  主持人设置考题状态为``开始''
\item
  开始状态下为考题设置题目属性
\item
  开始状态下为考题设置相关管理人员
\item
  主持人为开始状态下的考题分配知识点
\item
  主持人设定编写考题与评审考题的时间限制
\end{itemize}

\hypertarget{ux63d0ux4ea4ux8003ux9898}{%
\subsubsection{提交考题}\label{ux63d0ux4ea4ux8003ux9898}}

\begin{itemize}
\tightlist
\item
  主持人完成所有考题设定后提交考题
\item
  主持人未完成章节或知识点设定提交考题
\item
  主持人未完成其他设定提交考题
\end{itemize}

\hypertarget{ux4feeux6539ux72b6ux6001}{%
\subsubsection{修改状态}\label{ux4feeux6539ux72b6ux6001}}

\begin{itemize}
\tightlist
\item
  主持人在创建考题后修改状态为编写
\end{itemize}

\hypertarget{ux7f16ux5199ux8003ux9898}{%
\subsection{编写考题}\label{ux7f16ux5199ux8003ux9898}}

\hypertarget{ux8bbeux7f6eux591aux7c7bux578bux8003ux9898}{%
\subsubsection{设置多类型考题}\label{ux8bbeux7f6eux591aux7c7bux578bux8003ux9898}}

\begin{itemize}
\tightlist
\item
  作者编写纯文字考题
\item
  作者编写带有表格的考题
\item
  作者编写带有图形的考题
\item
  作者以Excel形式将考题导入出题系统
\item
  作者以XML形式将考题导入出题系统
\end{itemize}

\hypertarget{ux4feeux6539ux72b6ux6001-1}{%
\subsubsection{修改状态}\label{ux4feeux6539ux72b6ux6001-1}}

\begin{itemize}
\tightlist
\item
  作者在编写过程中设置题目状态为``评审''
\item
  作者在完成编写后把题目状态设置为``评审''
\end{itemize}

\hypertarget{ux8bc4ux5ba1ux8003ux9898}{%
\subsection{评审考题}\label{ux8bc4ux5ba1ux8003ux9898}}

\hypertarget{ux8bc4ux5ba1ux53efux63a5ux53d7}{%
\subsubsection{评审可接受}\label{ux8bc4ux5ba1ux53efux63a5ux53d7}}

\begin{itemize}
\tightlist
\item
  评审员设置评审结果为``可接受''
\end{itemize}

\hypertarget{ux8bc4ux5ba1ux88abux62d2ux7edd}{%
\subsubsection{评审被拒绝}\label{ux8bc4ux5ba1ux88abux62d2ux7edd}}

\begin{itemize}
\tightlist
\item
  评审员设置评审结果为``被拒绝''
\end{itemize}

\hypertarget{ux8bc4ux5ba1ux9700ux4feeux6539}{%
\subsubsection{评审需修改}\label{ux8bc4ux5ba1ux9700ux4feeux6539}}

\begin{itemize}
\tightlist
\item
  评审员设置评审结果为``需修改''
\end{itemize}

\hypertarget{ux8bc4ux5ba1ux6743ux9650ux63a7ux5236}{%
\subsubsection{评审权限控制}\label{ux8bc4ux5ba1ux6743ux9650ux63a7ux5236}}

\begin{itemize}
\tightlist
\item
  评审员阅读分配属于自己评审的考题
\item
  评审员对属于自己评审的考题编写评审意见和建议
\item
  评审员改变属于自己评审的考题的状态
\item
  主持人阅读该考题
\item
  主持人修改和操作该考题
\end{itemize}

\hypertarget{ux518dux5ba1ux8003ux9898}{%
\subsection{再审考题}\label{ux518dux5ba1ux8003ux9898}}

\hypertarget{ux518dux5ba1ux53efux53d1ux5e03}{%
\subsubsection{再审可发布}\label{ux518dux5ba1ux53efux53d1ux5e03}}

\begin{itemize}
\tightlist
\item
  评审员设置题目状态为``发布''
\end{itemize}

\hypertarget{ux518dux5ba1ux9700ux4feeux6539}{%
\subsubsection{再审需修改}\label{ux518dux5ba1ux9700ux4feeux6539}}

\begin{itemize}
\tightlist
\item
  评审员设置评审结果为``修改''
\end{itemize}

\hypertarget{ux518dux5ba1ux9700ux4f5cux5e9f}{%
\subsubsection{再审需作废}\label{ux518dux5ba1ux9700ux4f5cux5e9f}}

\begin{itemize}
\tightlist
\item
  评审员设置评审结果为``作废''
\end{itemize}

\hypertarget{ux518dux5ba1ux6743ux9650ux63a7ux5236}{%
\subsubsection{再审权限控制}\label{ux518dux5ba1ux6743ux9650ux63a7ux5236}}

\begin{itemize}
\tightlist
\item
  主持人修改考题
\item
  主持人改变考题状态
\end{itemize}

\hypertarget{ux4feeux6539ux8003ux9898}{%
\subsection{修改考题}\label{ux4feeux6539ux8003ux9898}}

\hypertarget{ux4feeux6539ux8003ux9898-1}{%
\subsubsection{修改考题}\label{ux4feeux6539ux8003ux9898-1}}

\begin{itemize}
\tightlist
\item
  作者阅读意见并修改考题
\item
  作者改变考题状态为``评审''
\end{itemize}

\hypertarget{ux4feeux6539ux6743ux9650ux63a7ux5236}{%
\subsubsection{修改权限控制}\label{ux4feeux6539ux6743ux9650ux63a7ux5236}}

\begin{itemize}
\tightlist
\item
  主持人修改考题
\item
  主持人改变考题状态
\end{itemize}

\hypertarget{ux53d1ux5e03ux8003ux9898}{%
\subsection{发布考题}\label{ux53d1ux5e03ux8003ux9898}}

\hypertarget{ux5bfcux51faux53d1ux5e03ux8003ux9898}{%
\subsubsection{导出发布考题}\label{ux5bfcux51faux53d1ux5e03ux8003ux9898}}

\begin{itemize}
\tightlist
\item
  主持人在考题处于``发布''状态下导出考题
\item
  主持人在考题处于``发布''状态下修改考题
\end{itemize}

\hypertarget{ux5bfcux5165ux9898ux5e93}{%
\subsubsection{导入题库}\label{ux5bfcux5165ux9898ux5e93}}

\begin{itemize}
\tightlist
\item
  题库系统管理员将已发布并且由主持人导出的考题文件导入题库
\end{itemize}

\hypertarget{ux4f5cux5e9fux8003ux9898}{%
\subsection{作废考题}\label{ux4f5cux5e9fux8003ux9898}}

\hypertarget{ux5e9fux9664ux4f5cux5e9fux8003ux9898}{%
\subsubsection{废除作废考题}\label{ux5e9fux9664ux4f5cux5e9fux8003ux9898}}

\begin{itemize}
\tightlist
\item
  主持人查看作废考题
\item
  主持人修改考题或改变状态
\end{itemize}

\hypertarget{ux91cdux542fux51faux9898ux6d41ux7a0b}{%
\subsubsection{重启出题流程}\label{ux91cdux542fux51faux9898ux6d41ux7a0b}}

\begin{itemize}
\tightlist
\item
  主持人查看作废考题后重新启动项目
\end{itemize}

\hypertarget{ux751fux6210ux8003ux5377}{%
\subsection{生成考卷}\label{ux751fux6210ux8003ux5377}}

\hypertarget{ux586bux5199ux8003ux8bd5ux540dux79f0}{%
\subsubsection{填写考试名称}\label{ux586bux5199ux8003ux8bd5ux540dux79f0}}

\begin{itemize}
\tightlist
\item
  输入合法的考试名称
\item
  输入非法的考试名称
\end{itemize}

\hypertarget{ux8bbeux7f6eux9898ux76eeux603bux6570}{%
\subsubsection{设置题目总数}\label{ux8bbeux7f6eux9898ux76eeux603bux6570}}

\begin{itemize}
\tightlist
\item
  输入范围为1\textasciitilde{}5的数字
\end{itemize}

\hypertarget{ux9009ux62e9ux8bd5ux5377ux96beux5ea6}{%
\subsubsection{选择试卷难度}\label{ux9009ux62e9ux8bd5ux5377ux96beux5ea6}}

\begin{itemize}
\tightlist
\item
  选择简单
\item
  选择一般
\item
  选择困难
\end{itemize}

\hypertarget{ux586bux5199ux5f00ux59cbux65f6ux95f4}{%
\subsubsection{填写开始时间}\label{ux586bux5199ux5f00ux59cbux65f6ux95f4}}

\begin{itemize}
\tightlist
\item
  填写格式正确的时间
\item
  填写格式错误的时间
\end{itemize}

\hypertarget{ux586bux5199ux7ed3ux675fux65f6ux95f4}{%
\subsubsection{填写结束时间}\label{ux586bux5199ux7ed3ux675fux65f6ux95f4}}

\begin{itemize}
\tightlist
\item
  填写格式正确的时间
\item
  填写格式错误的时间
\end{itemize}

\hypertarget{ux9009ux62e9ux7ae0ux8282}{%
\subsubsection{选择章节}\label{ux9009ux62e9ux7ae0ux8282}}

\begin{itemize}
\tightlist
\item
  选择具体的章节
\end{itemize}

\hypertarget{ux9009ux62e9ux8003ux8bd5ux5b66ux751f}{%
\subsubsection{选择考试学生}\label{ux9009ux62e9ux8003ux8bd5ux5b66ux751f}}

\begin{itemize}
\tightlist
\item
  选择本次考试的学生
\end{itemize}

\hypertarget{ux8003ux8bd5}{%
\subsection{考试}\label{ux8003ux8bd5}}

\hypertarget{ux67e5ux770bux8003ux5377}{%
\subsubsection{查看考卷}\label{ux67e5ux770bux8003ux5377}}

\begin{itemize}
\tightlist
\item
  点击侧边栏查看考卷
\end{itemize}

\hypertarget{ux53c2ux52a0ux8003ux8bd5}{%
\subsubsection{参加考试}\label{ux53c2ux52a0ux8003ux8bd5}}

\begin{itemize}
\tightlist
\item
  点击图标开始开始
\item
  点击考题答案的单选框进行选择
\end{itemize}

\hypertarget{ux63d0ux4ea4ux8003ux5377}{%
\subsubsection{提交考卷}\label{ux63d0ux4ea4ux8003ux5377}}

\begin{itemize}
\tightlist
\item
  回答完所有题目提交考卷
\item
  没有答完所有题目提交考卷
\end{itemize}

\hypertarget{ux67e5ux770bux5206ux6570}{%
\subsubsection{查看分数}\label{ux67e5ux770bux5206ux6570}}

\begin{itemize}
\tightlist
\item
  点击查看成绩
\item
  没有参加考试时点击查看成绩
\end{itemize}

\hypertarget{ux6d4bux8bd5ux8303ux56f4ux5236ux5b9a}{%
\section{测试范围制定}\label{ux6d4bux8bd5ux8303ux56f4ux5236ux5b9a}}

\hypertarget{ux51faux9898ux573aux666f}{%
\subsection{出题场景}\label{ux51faux9898ux573aux666f}}

\hypertarget{ux767bux5f55-1}{%
\subsubsection{登录}\label{ux767bux5f55-1}}

\begin{itemize}
\tightlist
\item
  登录
\end{itemize}

\hypertarget{ux5efaux7acbux65b0ux9879ux76ee-1}{%
\subsubsection{建立新项目}\label{ux5efaux7acbux65b0ux9879ux76ee-1}}

\begin{itemize}
\tightlist
\item
  创建项目
\end{itemize}

\hypertarget{ux65b0ux5efaux8003ux9898-1}{%
\subsubsection{新建考题}\label{ux65b0ux5efaux8003ux9898-1}}

\begin{itemize}
\tightlist
\item
  新建考题
\end{itemize}

\hypertarget{ux7f16ux5199ux8003ux9898-1}{%
\subsubsection{编写考题}\label{ux7f16ux5199ux8003ux9898-1}}

\begin{itemize}
\tightlist
\item
  设置多类型考题
\item
  修改状态
\end{itemize}

\hypertarget{ux8bc4ux5ba1ux8003ux9898-1}{%
\subsubsection{评审考题}\label{ux8bc4ux5ba1ux8003ux9898-1}}

\begin{itemize}
\tightlist
\item
  评审可接受
\item
  评审被拒绝
\item
  评审需修改
\item
  评审权限控制
\end{itemize}

\hypertarget{ux518dux5ba1ux8003ux9898-1}{%
\subsubsection{再审考题}\label{ux518dux5ba1ux8003ux9898-1}}

\begin{itemize}
\tightlist
\item
  再审可接受
\item
  再审被拒绝
\item
  再审需修改
\item
  再审权限控制
\end{itemize}

\hypertarget{ux53d1ux5e03ux8003ux9898-1}{%
\subsubsection{发布考题}\label{ux53d1ux5e03ux8003ux9898-1}}

\begin{itemize}
\tightlist
\item
  导出发布考题
\item
  导入题库
\end{itemize}

\hypertarget{ux4f5cux5e9fux8003ux9898-1}{%
\subsubsection{作废考题}\label{ux4f5cux5e9fux8003ux9898-1}}

\begin{itemize}
\tightlist
\item
  废除作废考题
\item
  重启出题流程
\end{itemize}

\hypertarget{ux521bux5efaux8bd5ux5377ux4e0eux8003ux8bd5ux573aux666f}{%
\subsection{创建试卷与考试场景}\label{ux521bux5efaux8bd5ux5377ux4e0eux8003ux8bd5ux573aux666f}}

\hypertarget{ux767bux5f55ux53caux4e2aux4ebaux4fe1ux606f-1}{%
\subsubsection{登录及个人信息}\label{ux767bux5f55ux53caux4e2aux4ebaux4fe1ux606f-1}}

\begin{itemize}
\tightlist
\item
  登录
\item
  修改个人信息
\end{itemize}

\hypertarget{ux751fux6210ux8003ux5377-1}{%
\subsubsection{生成考卷}\label{ux751fux6210ux8003ux5377-1}}

\begin{itemize}
\tightlist
\item
  填写考试名称
\item
  设置题目总数
\item
  选择试卷难度
\item
  填写开始时间
\item
  填写结束时间
\item
  选择章节
\item
  选择考试学生
\end{itemize}

\hypertarget{ux53c2ux52a0ux8003ux8bd5ux573aux666f}{%
\subsection{参加考试场景}\label{ux53c2ux52a0ux8003ux8bd5ux573aux666f}}

\hypertarget{ux767bux5f55-2}{%
\subsubsection{登录}\label{ux767bux5f55-2}}

\begin{itemize}
\tightlist
\item
  登录
\end{itemize}

\hypertarget{ux67e5ux770bux8003ux5377-1}{%
\subsubsection{查看考卷}\label{ux67e5ux770bux8003ux5377-1}}

\begin{itemize}
\tightlist
\item
  查看考卷
\end{itemize}

\hypertarget{ux53c2ux52a0ux8003ux8bd5-1}{%
\subsubsection{参加考试}\label{ux53c2ux52a0ux8003ux8bd5-1}}

\begin{itemize}
\tightlist
\item
  参加考试
\item
  开始考试
\end{itemize}

\hypertarget{ux63d0ux4ea4ux8003ux5377-1}{%
\subsubsection{提交考卷}\label{ux63d0ux4ea4ux8003ux5377-1}}

\begin{itemize}
\tightlist
\item
  提交考卷
\end{itemize}

\hypertarget{ux67e5ux770bux5206ux6570-1}{%
\subsubsection{查看分数}\label{ux67e5ux770bux5206ux6570-1}}

\begin{itemize}
\tightlist
\item
  查看分数
\end{itemize}

\hypertarget{ux6d4bux8bd5ux7ed3ux679cux6574ux7406}{%
\section{测试结果整理}\label{ux6d4bux8bd5ux7ed3ux679cux6574ux7406}}

\hypertarget{bugux7b49ux7ea7ux8bc4ux4f30}{%
\section{bug等级评估}\label{bugux7b49ux7ea7ux8bc4ux4f30}}

\hypertarget{ux603bux7ed3ux9648ux8bcd}{%
\section{总结陈词}\label{ux603bux7ed3ux9648ux8bcd}}

\hypertarget{ux51faux9898ux573aux666f-1}{%
\subsection{出题场景}\label{ux51faux9898ux573aux666f-1}}

\hypertarget{ux6d4bux8bd5ux6d41ux7a0b}{%
\subsubsection{测试流程}\label{ux6d4bux8bd5ux6d41ux7a0b}}

出题场景涉及多类参与人员以及双系统(管理系统与用户系统),测试的流程会比较复杂。本场景测试中对于出题场景的核心测试是从创建项目开始,到考题被设定为发布或者废弃结束。一个关于成功出题的测试流程如下:

\begin{enumerate}
\def\labelenumi{\arabic{enumi}.}
\item
  人员登录
\item
  主持人创建项目
\item
  主持人新建考题
\item
  考题作者编写考题
\item
  考题评审员评审考题。根据评审结果,可以回到流程4。
\item
  考题质管再审考题。根据评审与再审结果,可以回到流程4。
\item
  发布考题
\end{enumerate}

出题的结果也可能是作废考题,它的出现是因为流程5、6出现了其他基本流。而系统中也会存在较多备选流,比如登录时输入错误用户名或密码,新建考题不完成考题设定,考题作者不编写考题企图修改状态等。备选流与基本流相对,也叫无效流或错误流,它模拟用户错误的业务操作流程。

\hypertarget{ux6d4bux8bd5ux7ed3ux679cux5206ux6790}{%
\subsubsection{测试结果分析}\label{ux6d4bux8bd5ux7ed3ux679cux5206ux6790}}

虽然关于编写、评审、再审考题等重要功能的测试基本上没有发现缺陷,关于用户(作者、评审员、质管、主持人)权限测试也没有发现缺陷,但是测试人员发现了一些严重的功能不足。所以根据测试结果,我们可以得出结论:出题模块暂时不能提交给客户验证。主要存在以下情况:

\begin{itemize}
\item
  出题模块存在严重功能缺陷。在测试导入功能时,测试人员未找到相应交互模块。在测试导出功能时,导出格式不符合要求,题目导出位置无法锁定。
\item
  出题模块部分功能存在一般错误。比如在创建考题测试中,根据产品规格说明,应该完成所有设定后方能提交,但实际测试过程中发现无需要设定章节和知识点可以成功创建考题。
\item
  出题模块部分功能存在轻微缺陷。如测试编写带有图片的考题时,页面会存在卡顿现象。部分错误操作的提示含义不明或过于复杂。系统操作后对于其他相关人员的邮件提醒或其他提醒无法锁定。
\item
  出题模块可能存在未知风险。出题模块的流程是可成环的,本场景测试只选取了部分测试路径,那么对于过长出题流程是否存在资源泄露、系统崩溃等问题需要进一步验证。出题系统中需要用户输入文本,本场景测试中未测试是否可以避免脚本攻击、SQL注入等安全性问题。
\end{itemize}

\hypertarget{ux7ec4ux5377ux573aux666f}{%
\subsection{组卷场景}\label{ux7ec4ux5377ux573aux666f}}

对于组卷部分,管理员在登录后,点击查看考卷进入到组卷页面。然后依次输入考试名称,选择题目总数、试卷难度,输入开始和结束时间,选择章节以及选择参与这场考试的学生,这些是完成一次成功的组卷所必须经过的步骤。在这部分操作之外,登录及个人信息部分测试了错误的用户名密码的输入,以及修改个人信息。组卷部分对输入非法名称,格式错误的开始和结束时间这两个容易发生的错误操作进行了测试。
首先,对于组卷功能涉及的一系列操作都已经实现,可以完成一次成功的组卷。对于输入的非法的考试名称(此处使用的是为空的名称进行测试),系统能够给出提示。所以系统在正确使用的前提下可以正常工作,对于操作过程中可能出现的错误,系统也进行了一定的错误处理。但个人信息修改因为功能包含在出题系统中,所以同样没有实现,以及在输入格式错误的时间时,系统给出的提示是http的状态500,对于用户来说可能无法立刻明白出错的原因,但通过修改为正确的格式可以完成组卷。所以系统中还包含了一些未实现的需求以及用户不友好的部分,但并不是致命性的错误。
系统其他可能存在的风险包括其他的非法输入以及安全性方面的风险。对于非法名称,只测试了为空,而没有测试重复的名称和过长的名称。对于章节和学生选择的多选框如果没有进行选择的情况没有进行测试,以及在没有登录的情况下是否能进入这一界面没有测试。

\hypertarget{ux8003ux8bd5ux573aux666f}{%
\subsection{考试场景}\label{ux8003ux8bd5ux573aux666f}}

考试部分的必经流程包括,学生登录,查看自己参加的所有考试,开始考试并获取试卷,点击单选框选择答案,提交考卷,最后查看自己分数。除了以上的步骤外,包括了使用错误的用户名和密码登录,在没有答完所有题目的情况下提交试卷,以及在没有参加考试时查看分数。
在操作正确的前提下,参加考试并获取分数这一系列的步骤可以正常完成。对于没有参加的考试,查看到的分数为零分。但在没有答完所有题目的情况下提交试卷,系统会给出错误提示,这与一般的处理方法并不一致。所以系统对于基本的操作流程,大体实现了相关功能,但对于一些特殊情况,系统的处理可能与常理不同,应当加以说明。
没有覆盖的风险包括与考试时间相关的错误操作以及安全性相关的风险,如在考试未开始或结束后试图开始考试,在考试结束前查看成绩,在没有登录的情况下进行除了登录以外的操作,以及尝试获取自己没有参加的考试的试卷。

\pagebreak

\hypertarget{ux53c2ux8003ux6587ux732e}{%
\section*{参考文献}\label{ux53c2ux8003ux6587ux732e}}
\addcontentsline{toc}{section}{参考文献}

\hypertarget{refs}{}
\leavevmode\hypertarget{ref-innovativeInternationalisation}{}%
International Organization for Standardization. 2014. \emph{Systems and
Software Engineering --- Systems and Software Quality Requirements and
Evaluation (SQuaRE) --- Guide to SQuaRE}. \emph{International
Organization for Standardization}. Vol. 2014.
\url{https://www.iso.org/standard/64764.html}.

\leavevmode\hypertarget{ref-innovative1}{}%
中国国家标准化管理委员会. 2016. \emph{GB/T
25000.51-2016《系统与软件工程系统与软件质量要求和评价 (SQuaRE) 第 51
部分 : 就绪可用软件产品 (RUSP) 的质量要求和测试细则》}.
\emph{系统与软件工程系统与软件质量要求和评价 (SQuaRE)}. Vol. 51.
中国国家标准化管理委员会. \url{http://openstd.samr.gov.cn}.

\leavevmode\hypertarget{ref-innovative3}{}%
---------. 2017a. \emph{GB/T 25000.12-2017《系统与软件工程
系统与软件质量要求和评价(SQuaRE) 第12部分:数据质量模型》}.
\emph{系统与软件工程系统与软件质量要求和评价 (SQuaRE)}. Vol. 12.
中国国家标准化管理委员会. \url{http://openstd.samr.gov.cn}.

\leavevmode\hypertarget{ref-innovative4}{}%
---------. 2017b. \emph{GB/T 25000.24-2017《系统与软件工程
系统与软件质量要求和评价(SQuaRE) 第24部分:数据质量测量》}.
\emph{系统与软件工程系统与软件质量要求和评价 (SQuaRE)}. Vol. 24.
中国国家标准化管理委员会. \url{http://openstd.samr.gov.cn}.

\leavevmode\hypertarget{ref-innovative5}{}%
---------. 2018. \emph{GB/T 25000.40-201《系统与软件工程
系统与软件质量要求和评价(SQuaRE) 第40部分:评价过程》}.
\emph{系统与软件工程系统与软件质量要求和评价 (SQuaRE)}. Vol. 40.
中国国家标准化管理委员会. \url{http://openstd.samr.gov.cn}.

\leavevmode\hypertarget{ref-innovative2}{}%
---------. 2019. \emph{GB/T 25000.23-2019《系统与软件工程
系统与软件质量要求和评价(SQuaRE) 第23部分:系统与软件产品质量测量》}.
\emph{系统与软件工程系统与软件质量要求和评价 (SQuaRE)}. Vol. 23.
中国国家标准化管理委员会. \url{http://openstd.samr.gov.cn}.

\end{document}
