\documentclass[hyperref, a4paper]{ctexart}
\usepackage{lmodern}
\usepackage{amssymb,amsmath}
\usepackage{ifxetex,ifluatex}
\usepackage{fixltx2e} % provides \textsubscript
\ifnum 0\ifxetex 1\fi\ifluatex 1\fi=0 % if pdftex
  \usepackage[T1]{fontenc}
  \usepackage[utf8]{inputenc}
\else % if luatex or xelatex
  \ifxetex
    \usepackage{xltxtra,xunicode}
  \else
    \usepackage{fontspec}
  \fi
  \defaultfontfeatures{Mapping=tex-text,Scale=MatchLowercase}
  \newcommand{\euro}{€}
\fi
% use upquote if available, for straight quotes in verbatim environments
\IfFileExists{upquote.sty}{\usepackage{upquote}}{}
% use microtype if available
\IfFileExists{microtype.sty}{%
\usepackage{microtype}
\UseMicrotypeSet[protrusion]{basicmath} % disable protrusion for tt fonts
}{}
\ifxetex
  \usepackage[setpagesize=false, % page size defined by xetex
              unicode=false, % unicode breaks when used with xetex
              xetex]{hyperref}
\else
  \usepackage[unicode=true]{hyperref}
\fi
\usepackage[usenames,dvipsnames]{color}
\hypersetup{breaklinks=true,
            bookmarks=true,
            pdfauthor={Tian, Jiahe; Hu, Xiaoxiao; Huang, Jiani; Liu, Jiaxing; Shi, Ruixin; Wu, Chenning; Zhang, Cenyuan; Zhang, Yihan; Wang, Chen},
            pdftitle={ 出题系统需求文档},
            colorlinks=true,
            citecolor=blue,
            urlcolor=blue,
            linkcolor=magenta,
            pdfborder={0 0 0}}
\urlstyle{same}  % don't use monospace font for urls
\usepackage{longtable,booktabs}
\setlength{\emergencystretch}{3em}  % prevent overfull lines
\providecommand{\tightlist}{%
  \setlength{\itemsep}{0pt}\setlength{\parskip}{0pt}}
\setcounter{secnumdepth}{5}

\title{\vspace{2in} 出题系统需求文档\\\vspace{0.5em}{\large 软件质量保障与测试课程Lab3课程作业(第9组)}}
\author{Tian, Jiahe\footnote{Equal Contribution, Fudan University, 17307130313
  (\href{mailto:tianjh17@fudan.edu.cn}{\nolinkurl{tianjh17@fudan.edu.cn}})} \and Hu, Xiaoxiao\footnote{Equal Contribution, Fudan University, 17302010077
  (\href{mailto:xxhu17@fudan.edu.cn}{\nolinkurl{xxhu17@fudan.edu.cn}})} \and Huang, Jiani\footnote{Equal Contribution, Fudan University, 17302010063
  (\href{mailto:huangjn17@fudan.edu.cn}{\nolinkurl{huangjn17@fudan.edu.cn}})} \and Liu, Jiaxing\footnote{Equal Contribution, Fudan University, 17302010049
  (\href{mailto:jiaxingliu17@fudan.edu.cn}{\nolinkurl{jiaxingliu17@fudan.edu.cn}})} \and Shi, Ruixin\footnote{Equal Contribution, Fudan University, 17302010065
  (\href{mailto:rxshi17@fudan.edu.cn}{\nolinkurl{rxshi17@fudan.edu.cn}})} \and Wu, Chenning\footnote{Equal Contribution, Fudan University, 17302010066
  (\href{mailto:cnwu17@fudan.edu.cn}{\nolinkurl{cnwu17@fudan.edu.cn}})} \and Zhang, Cenyuan\footnote{Equal Contribution, Fudan University,
  17302010068
  (\href{mailto:cenyuanzhang17@fudan.edu.cn}{\nolinkurl{cenyuanzhang17@fudan.edu.cn}})} \and Zhang, Yihan\footnote{Equal Contribution, Fudan University, 17302010076
  (\href{mailto:zhangyihan17@fudan.edu.cn}{\nolinkurl{zhangyihan17@fudan.edu.cn}})} \and Wang, Chen\footnote{Equal Contribution, Fudan University, 16307110064
  (\href{mailto:wangc16@fudan.edu.cn}{\nolinkurl{wangc16@fudan.edu.cn}})}}
\date{2020年4月1日}



% Redefines (sub)paragraphs to behave more like sections
\ifx\paragraph\undefined\else
\let\oldparagraph\paragraph
\renewcommand{\paragraph}[1]{\oldparagraph{#1}\mbox{}}
\fi
\ifx\subparagraph\undefined\else
\let\oldsubparagraph\subparagraph
\renewcommand{\subparagraph}[1]{\oldsubparagraph{#1}\mbox{}}
\fi

\begin{document}
\maketitle

\newpage

\LARGE

\begin{center}
\textbf{出题系统需求文档}
\end{center}

\large
\begin{center}
\textbf{\emph{软件质量保障与测试课程Lab3课程作业}}
\end{center}

\hypertarget{ux6458ux8981}{%
\section*{摘要}\label{ux6458ux8981}}
\addcontentsline{toc}{section}{摘要}

本次作业为软件质量保障与测试课程的Lab3课程作业,需要我们以小组为单位撰写CS选课系统的IEEE829测试文档.

\hypertarget{ux5173ux952eux8bcd}{%
\section*{关键词}\label{ux5173ux952eux8bcd}}
\addcontentsline{toc}{section}{关键词}

系统与软件工程; 系统与软件质量要求和评价; 需求文档

\normalsize

\newpage

\tableofcontents

\newpage

\hypertarget{ux76eeux7684ux5220ux9664}{%
\section{目的(删除)}\label{ux76eeux7684ux5220ux9664}}

本测试计划基于Lab2版本的CS出题系统,本测试文档将涵盖以下方面: -
定义测试过程所需的测试方法与工具 -
与相关人员沟通测试任务分派,确定测试时间表、测试环境与风险控制 -
定义测试进行的具体步骤

\hypertarget{ux6d4bux8bd5ux8ba1ux5212ux6807ux8bc6}{%
\section{测试计划标识}\label{ux6d4bux8bd5ux8ba1ux5212ux6807ux8bc6}}

CSTQB4.0-TP1.0 修订历史

\begin{tabular}{|p{2cm}|p{3.3cm}|p{5cm}|}
\hline
版本 & 日期 & 说明\\
\hline
草稿 & 2021 年3 月7 日 & 基于出题系统的测试计划草稿\\
\hline
CSTQB4.0-TP1.0 & 2021 年4 月1 日 & 建立更完善的出题系统的测试计划\\
\hline
\end{tabular}

\hypertarget{ux4ecbux7ecd}{%
\section{介绍}\label{ux4ecbux7ecd}}

\hypertarget{ux76eeux6807}{%
\subsection{目标}\label{ux76eeux6807}}

本测试计划基于Lab2版本的CS出题系统,本测试文档将涵盖以下方面: -
定义测试过程所需的测试方法与工具 -
与相关人员沟通测试任务分派,确定测试时间表、测试环境与风险控制 -
定义测试进行的具体步骤 \#\# 测试软件情况介绍:
出题系统是在线出题考试系统的的一个子系统。利用出题系统,出题专家们可以方便的进行网上考题编写和评审,并无缝的与在线出题考试系统进行衔接,提高考题编写的质量和效率,也便于考题的统一管理和评估。

\hypertarget{ux6d4bux8bd5ux9879}{%
\section{测试项}\label{ux6d4bux8bd5ux9879}}

\hypertarget{ux767bux9646ux53caux4e2aux4ebaux4fe1ux606f}{%
\subsection{登陆及个人信息}\label{ux767bux9646ux53caux4e2aux4ebaux4fe1ux606f}}

\hypertarget{ux7528ux6237ux767bux5f55ux6821ux9a8c}{%
\subsubsection{用户登录校验}\label{ux7528ux6237ux767bux5f55ux6821ux9a8c}}

\begin{itemize}
\tightlist
\item
  先决条件:用户已经成功在系统中注册过自己的用户名和密码

  \begin{itemize}
  \tightlist
  \item
    用户输入正确用户名和密码可以成功登录,登录状态由``未登录''变为对应用户名。
  \item
    用户输入错误用户名和密码无法登陆系统,提示``用户名或密码''错误,允许继续输入,不允许对考题进行任一操作。
  \end{itemize}
\end{itemize}

\hypertarget{ux7528ux6237ux4feeux6539ux4e2aux4ebaux4fe1ux606f}{%
\subsubsection{用户修改个人信息}\label{ux7528ux6237ux4feeux6539ux4e2aux4ebaux4fe1ux606f}}

\begin{itemize}
\tightlist
\item
  先决条件:用户登陆系统成功

  \begin{itemize}
  \tightlist
  \item
    用户可以对用户信息(包括用户名、密码、其他信息等)进行修改,修改完成后自动更新用户信息。
  \end{itemize}
\end{itemize}

\hypertarget{ux5efaux7acbux65b0ux9879ux76ee}{%
\subsection{建立新项目}\label{ux5efaux7acbux65b0ux9879ux76ee}}

\hypertarget{ux4e3bux6301ux4ebaux521bux5efaux9879ux76ee}{%
\subsubsection{主持人创建项目}\label{ux4e3bux6301ux4ebaux521bux5efaux9879ux76ee}}

\begin{itemize}
\tightlist
\item
  先决条件:登陆用户角色为主持人

  \begin{itemize}
  \tightlist
  \item
    主持人可以创建新的出题系统,一旦创建不允许更换其他主持人。
  \item
    主持人在创建时定义项目名,不能与其他出题系统重复,不允许创建完成后再修改项目名。
  \item
    主持人在创建项目时可以成功对项目进行规划,如限制项目考题数量或考题范围。
  \item
    系统在主持人成功创建项目并完成项目设置后自动发送邮件提示主持人。
  \end{itemize}
\end{itemize}

\hypertarget{ux4e3bux6301ux4ebaux65b0ux5efaux8003ux9898}{%
\subsubsection{主持人新建考题}\label{ux4e3bux6301ux4ebaux65b0ux5efaux8003ux9898}}

\begin{itemize}
\tightlist
\item
  先决条件:登陆用户角色为主持人且完成项目创建

  \begin{itemize}
  \tightlist
  \item
    主持人可以创建新考题,成功创建的考题有唯一不重复ID。
  \item
    主持人可以设置新考题状态为``开始''。
  \item
    主持人在开始状态下可以为考题设置题目属性(如知识点,时间点、时间安排)
  \item
    主持人在开始状态下可以为考题设置相关管理人员,包括作者、评审员、质管员。其中不同身份不能安排同一名用户,如作者与评审员不能有同一个用户ID。每个身份要求至少安排一人,允许为一道考题安排多名作者、评审员、质管员。
  \item
    考题相关的评审员和作者之间可以互相读取权限信息。
  \item
    系统在主持人成功新建题目、设置题目属性与作者、评审员、质管员后会自动发送邮件给对应管理人员进行确认。
  \end{itemize}
\end{itemize}

\hypertarget{ux5f00ux59cbux542fux52a8ux9879ux76eeux72b6ux6001-ux5f00ux59cb-ux89d2ux8272-ux4e3bux6301ux4eba}{%
\subsection{开始启动项目(状态: 开始; 角色:
主持人)}\label{ux5f00ux59cbux542fux52a8ux9879ux76eeux72b6ux6001-ux5f00ux59cb-ux89d2ux8272-ux4e3bux6301ux4eba}}

\hypertarget{ux4e3bux6301ux4ebaux8bbeux7f6eux8003ux9898ux671fux9650}{%
\subsubsection{主持人设置考题期限}\label{ux4e3bux6301ux4ebaux8bbeux7f6eux8003ux9898ux671fux9650}}

\begin{itemize}
\tightlist
\item
  先决条件:登陆用户角色为主持人且已设置考题状态为``开始''

  \begin{itemize}
  \tightlist
  \item
    主持人能够为状态为``开始''的考题设定编写考题与评审考题的时间限期。
  \item
    系统会对超出限期的任务报警,自动发送邮件或站内信给主持人。
  \item
    主持人能对超出限期的任务提出警告或延长时间限期。
  \end{itemize}
\end{itemize}

\hypertarget{ux4e3bux6301ux4ebaux4feeux6539ux72b6ux6001ux5f00ux59cb-ux7f16ux5199}{%
\subsubsection{主持人修改状态(开始-编写)}\label{ux4e3bux6301ux4ebaux4feeux6539ux72b6ux6001ux5f00ux59cb-ux7f16ux5199}}

\begin{itemize}
\tightlist
\item
  先决条件:登陆用户角色为主持人且已设置考题状态为``开始''

  \begin{itemize}
  \tightlist
  \item
    主持人在开始状态下完成了对题目属性的设置、知识点的分配、作者、评审员和质管员的设置后,可以将题目状态改为``编写''。
  \item
    系统在题目状态改变为``编写''后能自动给对应作者发送提示邮件。
  \item
    主持人未完成相关设置的题目不能修改状态。
  \item
    主持人能够批量将完成分配的题目状态改为``编写'',系统会给对应作者发送邮件
  \item
    在考题状态为``开始''下,除主持人外,系统管理员、作者、评审员没有修改题目状态的权限。
  \end{itemize}
\end{itemize}

\hypertarget{ux7f16ux5199ux8003ux9898ux72b6ux6001-ux7f16ux5199-ux89d2ux8272-ux4f5cux8005}{%
\subsection{编写考题(状态: 编写; 角色:
作者)}\label{ux7f16ux5199ux8003ux9898ux72b6ux6001-ux7f16ux5199-ux89d2ux8272-ux4f5cux8005}}

\hypertarget{ux4f5cux8005ux8bbeux7f6eux591aux7c7bux578bux8003ux9898}{%
\subsubsection{作者设置多类型考题}\label{ux4f5cux8005ux8bbeux7f6eux591aux7c7bux578bux8003ux9898}}

\begin{itemize}
\tightlist
\item
  先决条件:登陆用户角色为作者并已被主持人分配出题任务

  \begin{itemize}
  \tightlist
  \item
    作者能够看到由主持人分配给自己编写的考题的相关信息。
  \item
    作者能够编写普通(纯文字)考题并设置相关属性。
  \item
    作者能够编写带有表格的题目并正确显示,无乱码并设置相关属性。
  \item
    作者能够编写带有图形的题目并正确显示,无乱码并设置相关属性。
  \item
    作者需要对题目给出标准答案并设置相关属性(如题型、知识点、分值等)。
  \item
    普通考题能以XML,Excel的形式导入出题系统
  \item
    带有表格的题目能以XML,Excel的形式导入出题系统
  \item
    带有图形的题目能以XML,Excel的形式导入出题系统
  \end{itemize}
\end{itemize}

\hypertarget{ux4f5cux8005ux4feeux6539ux72b6ux6001ux7f16ux5199-ux8bc4ux5ba1}{%
\subsubsection{作者修改状态(编写-评审)}\label{ux4f5cux8005ux4feeux6539ux72b6ux6001ux7f16ux5199-ux8bc4ux5ba1}}

\begin{itemize}
\tightlist
\item
  先决条件:登陆用户角色为作者并已被主持人分配出题任务

  \begin{itemize}
  \tightlist
  \item
    作者不可以对评审员、质管员进行修改,也不可以修改题目的编写时间限制或所属项目。仅允许且必须对题干、答案和部分题目属性进行编写和设置。
  \item
    对未完成编写(如题干为空等情况),未给出标准答案或未完成相关属性设定的题目,作者不能修改其状态为``评审''。
  \item
    作者能够将编写完成、给出标准答案并设置相关属性的题目状态改为``评审''。
  \item
    系统在题目状态改为``评审''后给评审员发出邮件。
  \item
    主持人在编写过程中没有修改和操作的权限。
  \end{itemize}
\end{itemize}

\hypertarget{ux8bc4ux5ba1ux8003ux9898ux72b6ux6001-ux8bc4ux5ba1-ux89d2ux8272-ux8bc4ux5ba1ux5458}{%
\subsection{评审考题(状态: 评审; 角色:
评审员)}\label{ux8bc4ux5ba1ux8003ux9898ux72b6ux6001-ux8bc4ux5ba1-ux89d2ux8272-ux8bc4ux5ba1ux5458}}

\hypertarget{ux8bc4ux5ba1ux5458ux8bbeux7f6eux8bc4ux5ba1ux7ed3ux679cux4e3aux53efux63a5ux53d7}{%
\subsubsection{评审员设置评审结果为``可接受''}\label{ux8bc4ux5ba1ux5458ux8bbeux7f6eux8bc4ux5ba1ux7ed3ux679cux4e3aux53efux63a5ux53d7}}

\begin{itemize}
\tightlist
\item
  先决条件:登陆用户角色为评审员并有需要此用户评审的考题

  \begin{itemize}
  \tightlist
  \item
    评审员能够设置评审结果为``可接受'',考题的状态从``评审''变为``再审''。
  \item
    系统在状态变化后通过邮件提醒相关质管员并提示评审信息。
  \end{itemize}
\end{itemize}

\hypertarget{ux8bc4ux5ba1ux5458ux8bbeux7f6eux8bc4ux5ba1ux7ed3ux679cux4e3aux88abux62d2ux7edd}{%
\subsubsection{评审员设置评审结果为``被拒绝''}\label{ux8bc4ux5ba1ux5458ux8bbeux7f6eux8bc4ux5ba1ux7ed3ux679cux4e3aux88abux62d2ux7edd}}

\begin{itemize}
\tightlist
\item
  先决条件:登陆用户角色为评审员并有需要此用户评审的考题

  \begin{itemize}
  \tightlist
  \item
    评审员能够设置评审结果为``被拒绝'',考题的状态从``评审''修改为``再审''。
  \item
    系统在状态变化后通过邮件提醒相关质管员并提示评审信息。
  \end{itemize}
\end{itemize}

\hypertarget{ux8bc4ux5ba1ux5458ux8bbeux7f6eux8bc4ux5ba1ux7ed3ux679cux4e3aux9700ux4feeux6539}{%
\subsubsection{评审员设置评审结果为``需修改''}\label{ux8bc4ux5ba1ux5458ux8bbeux7f6eux8bc4ux5ba1ux7ed3ux679cux4e3aux9700ux4feeux6539}}

\begin{itemize}
\tightlist
\item
  先决条件:登陆用户角色为评审员并有需要此用户评审的考题

  \begin{itemize}
  \tightlist
  \item
    评审员能够设置评审结果为``需修改'',考题的状态从``评审''变为``修改''。
  \item
    系统在状态变化后通过包含评审信息的邮件提醒相关作者修改。
  \end{itemize}
\end{itemize}

\hypertarget{ux8bc4ux5ba1ux5458ux8bc4ux5ba1ux6743ux9650}{%
\subsubsection{评审员评审权限}\label{ux8bc4ux5ba1ux5458ux8bc4ux5ba1ux6743ux9650}}

\begin{itemize}
\tightlist
\item
  先决条件:登陆用户角色为评审员,考题处于``评审''状态

  \begin{itemize}
  \tightlist
  \item
    评审员有权阅读被主持人分配属于自己评审的考题
  \item
    评审员有权对被主持人分配属于自己评审的考题编写评审意见和建议(包括指出考题的错误)
  \item
    评审员无权改变被主持人分配属于自己评审的考题(包括内容,属性及其他)
  \item
    评审员有权改变被主持人分配属于自己评审的考题的状态
  \item
    评审员无权阅读其他考题
  \item
    评审员无权对其他考题编写评审意见和建议(包括指出考题的错误)
  \item
    评审员无权改变其他考题(包括内容,属性及其他)
  \item
    评审员无权改变其他考题的状态
  \item
    主持人否有权阅读考题
  \item
    主持人无权对考题编写评审意见和建议(包括指出考题的错误)
  \item
    主持人无权改变考题(包括内容,属性及其他)
  \item
    主持人无权改变考题的状态
  \item
    其他角色无权阅读考题
  \item
    其他角色无权对考题编写评审意见和建议(包括指出考题的错误)
  \item
    其他角色无权改变考题(包括内容,属性及其他)
  \item
    其他角色无权改变考题的状态
  \end{itemize}
\end{itemize}

\hypertarget{ux518dux5ba1ux8003ux9898ux72b6ux6001-ux518dux5ba1-ux89d2ux8272-ux8d28ux7ba1ux5458}{%
\subsection{再审考题(状态: 再审; 角色:
质管员)}\label{ux518dux5ba1ux8003ux9898ux72b6ux6001-ux518dux5ba1-ux89d2ux8272-ux8d28ux7ba1ux5458}}

\hypertarget{ux8d28ux7ba1ux5458ux8bbeux7f6eux518dux5ba1ux7ed3ux679cux4e3aux53efux53d1ux5e03}{%
\subsubsection{质管员设置再审结果为``可发布''}\label{ux8d28ux7ba1ux5458ux8bbeux7f6eux518dux5ba1ux7ed3ux679cux4e3aux53efux53d1ux5e03}}

\begin{itemize}
\tightlist
\item
  先决条件:登陆用户角色为质管员并有需要此用户再审的考题

  \begin{itemize}
  \tightlist
  \item
    质管员能够设置再审结果为``可发布'',考题的状态从``再审''变为``发布''。
  \item
    系统在状态改变后能够通过邮件提醒主持人。
  \end{itemize}
\end{itemize}

\hypertarget{ux8d28ux7ba1ux5458ux8bbeux7f6eux518dux5ba1ux7ed3ux679cux4e3aux9700ux4feeux6539}{%
\subsubsection{质管员设置再审结果为``需修改''}\label{ux8d28ux7ba1ux5458ux8bbeux7f6eux518dux5ba1ux7ed3ux679cux4e3aux9700ux4feeux6539}}

\begin{itemize}
\tightlist
\item
  先决条件:登陆用户角色为质管员并有需要此用户再审的考题

  \begin{itemize}
  \tightlist
  \item
    质管员能够设置再审结果为``需修改'',考题的状态从``再审''变为``修改''。
  \item
    系统在状态改变后能够通过邮件提醒作者。
  \end{itemize}
\end{itemize}

\hypertarget{ux8d28ux7ba1ux5458ux8bbeux7f6eux518dux5ba1ux7ed3ux679cux4e3aux9700ux4f5cux5e9f}{%
\subsubsection{质管员设置再审结果为``需作废''}\label{ux8d28ux7ba1ux5458ux8bbeux7f6eux518dux5ba1ux7ed3ux679cux4e3aux9700ux4f5cux5e9f}}

\begin{itemize}
\tightlist
\item
  先决条件:登陆用户角色为质管员并有需要此用户再审的考题

  \begin{itemize}
  \tightlist
  \item
    质管员能够设置再审结果为``需作废'',考题的状态从``再审''变为``作废''。
  \item
    系统在状态改变后能够通过邮件提醒主持人。
  \end{itemize}
\end{itemize}

\hypertarget{ux8d28ux7ba1ux5458ux518dux5ba1ux6743ux9650}{%
\subsubsection{质管员再审权限}\label{ux8d28ux7ba1ux5458ux518dux5ba1ux6743ux9650}}

\begin{itemize}
\tightlist
\item
  先决条件:登陆用户角色为质管员,考题处于``再审''状态

  \begin{itemize}
  \tightlist
  \item
    质管员有权阅读考题
  \item
    质管员有权对考题编写评审意见和建议(包括指出考题的错误)
  \item
    质管员无权改变考题(包括内容,属性及其他)
  \item
    主持人有权阅读考题
  \item
    主持人无权对考题编写评审意见和建议(包括指出考题的错误)
  \item
    主持人无权改变考题(包括内容,属性及其他)
  \item
    主持人无权改变考题的状态
  \item
    其余角色无权阅读考题
  \item
    其余角色无权对考题编写评审意见和建议(包括指出考题的错误)
  \item
    其余角色无权改变考题(包括内容,属性及其他)
  \item
    其余角色无权改变考题的状态
  \end{itemize}
\end{itemize}

\hypertarget{ux4feeux6539ux8003ux9898ux72b6ux6001-ux4feeux6539-ux89d2ux8272-ux4f5cux8005}{%
\subsection{修改考题(状态: 修改; 角色:
作者)}\label{ux4feeux6539ux8003ux9898ux72b6ux6001-ux4feeux6539-ux89d2ux8272-ux4f5cux8005}}

\hypertarget{ux4f5cux8005ux4feeux6539ux8003ux9898}{%
\subsubsection{作者修改考题}\label{ux4f5cux8005ux4feeux6539ux8003ux9898}}

\begin{itemize}
\tightlist
\item
  先决条件:登陆用户角色为作者,考题处于修改状态

  \begin{itemize}
  \tightlist
  \item
    作者能够阅读修改要求及意见。
  \item
    作者将题目状态设置为``评审''后,系统自动给评审员发送状态变化的邮件提醒。
  \end{itemize}
\end{itemize}

\hypertarget{ux4f5cux8005ux4feeux6539ux6743ux9650}{%
\subsubsection{作者修改权限}\label{ux4f5cux8005ux4feeux6539ux6743ux9650}}

\begin{itemize}
\tightlist
\item
  先决条件:考题处于``修改''状态

  \begin{itemize}
  \tightlist
  \item
    作者能够阅读所有需要自己修改的考题
  \item
    作者可以对需要自己修改的考题(包括内容及属性)进行改动
  \item
    作者无权对评审员,质管员,编写时间限制,所属项目等进行修改
  \item
    作者可以改变处于``修改''且需要自己修改的考题的状态从``修改''到``评审''而非其他状态
  \item
    主持人能够阅读考题
  \item
    主持人无权对这些考题(包括内容,属性及其他)进行改动
  \item
    主持人无权改变考题的状态从``修改''到其他状态
  \item
    作者不能够阅读不需要自己修改的考题
  \item
    作者无权对不需要自己修改的考题(包括内容,属性及其他)进行改动
  \item
    作者无权改变处于``修改''状态且不需要自己修改的考题的状态从``修改''到``评审''
  \item
    作者无权改变处于``修改''状态且不需要自己修改的考题的状态从``修改''到其他状态
  \item
    其余角色不能够阅读考题
  \item
    其余角色无权对这些考题(包括内容,属性及其他)进行改动
  \item
    其余角色无权改变考题的状态从``修改''到其他状态
  \end{itemize}
\end{itemize}

\hypertarget{ux53d1ux5e03ux8003ux9898ux72b6ux6001-ux53d1ux5e03-ux89d2ux8272-ux4e3bux6301ux4eba}{%
\subsection{发布考题(状态: 发布; 角色:
主持人)}\label{ux53d1ux5e03ux8003ux9898ux72b6ux6001-ux53d1ux5e03-ux89d2ux8272-ux4e3bux6301ux4eba}}

\hypertarget{ux4e3bux6301ux4ebaux5bfcux51faux53d1ux5e03ux8003ux9898}{%
\subsubsection{主持人导出发布考题}\label{ux4e3bux6301ux4ebaux5bfcux51faux53d1ux5e03ux8003ux9898}}

\begin{itemize}
\tightlist
\item
  先决条件:考题处于``发布''状态

  \begin{itemize}
  \tightlist
  \item
    主持人可以阅读这些考题
  \item
    主持人可以根据考题导出规则导出考题
  \item
    主持人不可以修改考题(包括内容,属性及其他)
  \item
    主持人不可以修改考题的状态
  \item
    导出的考题需要符合考题导出规则
  \item
    导出的考题包括考题最终发布版本的所有完整内容及属性,或者同时导出考题的历史数据
  \item
    导出的考题文件可以为Excel或XML格式
  \item
    导出的考题中不含有冗余信息
  \item
    其余角色不可以阅读这些考题
  \item
    其余角色不可以导出考题
  \item
    其余角色不可以修改考题(包括内容,属性及其他)
  \item
    其余角色不可以修改考题的状态
  \end{itemize}
\end{itemize}

\hypertarget{ux4e3bux6301ux4ebaux5bfcux5165ux9898ux5e93}{%
\subsubsection{主持人导入题库}\label{ux4e3bux6301ux4ebaux5bfcux5165ux9898ux5e93}}

\begin{itemize}
\tightlist
\item
  先决条件:考题已发布,并由主持人以XML或Excel形式导出

  \begin{itemize}
  \tightlist
  \item
    题库系统管理员可以将符合以上先决条件的XML文件导入题库
  \item
    题库系统管理员可以将符合以上先决条件的Excel文件导入题库
  \item
    题库系统管理员不可以将未发布的考题导入题库
  \item
    题库系统管理员不可以将未经主持人导出的考题导入题库
  \item
    题库系统管理员不可以将非合法考题形式的XML或Excel文件导入题库
  \item
    其他角色不可以将任意XML或Excel文件导入题库
  \end{itemize}
\end{itemize}

\hypertarget{ux4f5cux5e9fux8003ux9898ux72b6ux6001-ux4f5cux5e9f-ux89d2ux8272-ux4e3bux6301ux4eba}{%
\subsection{作废考题(状态: 作废; 角色:
主持人)}\label{ux4f5cux5e9fux8003ux9898ux72b6ux6001-ux4f5cux5e9f-ux89d2ux8272-ux4e3bux6301ux4eba}}

\hypertarget{ux4e3bux6301ux4ebaux5e9fux9664ux4f5cux5e9fux8003ux9898}{%
\subsubsection{主持人废除作废考题}\label{ux4e3bux6301ux4ebaux5e9fux9664ux4f5cux5e9fux8003ux9898}}

\begin{itemize}
\tightlist
\item
  先决条件:考题处于作废状态

  \begin{itemize}
  \tightlist
  \item
    主持可以废除考题
  \item
    废除后的考题不能被使用(包括阅读及修改)
  \item
    主持人不能修改考题(包括内容,属性及其他)
  \item
    主持人不能修改考题的状态
  \item
    其余角色不能查看考题
  \item
    其余角色没有废除考题的权限
  \item
    其余角色不能修改考题(包括内容,属性及其他)
  \item
    其余角色不能修改考题的状态
  \end{itemize}
\end{itemize}

\hypertarget{ux4e3bux6301ux4ebaux91cdux542fux51faux9898ux6d41ux7a0b}{%
\subsubsection{主持人重启出题流程}\label{ux4e3bux6301ux4ebaux91cdux542fux51faux9898ux6d41ux7a0b}}

\begin{itemize}
\tightlist
\item
  先决条件:主持人对于处于作废状态的考题进行判断后,发现无法达成出题目的和要求

  \begin{itemize}
  \tightlist
  \item
    主持人拥有重启项目的权限
  \item
    主持人重启项目后,旧项目中的考题不能被阅读和修改
  \item
    主持人重启项目后,新项目中的考题能够正常编写及被评审
  \end{itemize}
\end{itemize}

\hypertarget{ux6d4bux8bd5ux9879ux901aux8fc7ux5931ux8d25ux51c6ux5219}{%
\section{测试项通过/失败准则}\label{ux6d4bux8bd5ux9879ux901aux8fc7ux5931ux8d25ux51c6ux5219}}

\begin{itemize}
\tightlist
\item
  组件测试相关准则

  \begin{itemize}
  \tightlist
  \item
    通过准则

    \begin{itemize}
    \tightlist
    \item
      关于该组件接口的准则通过率达100\%
    \item
      代码覆盖率达80\%以上
    \end{itemize}
  \item
    失败准则

    \begin{itemize}
    \tightlist
    \item
      任意测试组件的相关准则有不符合的条目即为失败
    \end{itemize}
  \end{itemize}
\item
  集成测试相关准则

  \begin{itemize}
  \tightlist
  \item
    通过准则

    \begin{itemize}
    \tightlist
    \item
      所有接口测试通过率达100\%
    \item
      单元测试平均代码覆盖率达80\%以上
    \end{itemize}
  \end{itemize}
\item
  系统测试相关准则

  \begin{itemize}
  \tightlist
  \item
    通过准则

    \begin{itemize}
    \tightlist
    \item
      该版本的系统功能能够完整运作
    \end{itemize}
  \end{itemize}
\item
  验收测试相关准则

  \begin{itemize}
  \tightlist
  \item
    通过准则

    \begin{itemize}
    \tightlist
    \item
      软件需求分析说明书中定义的所有功能已全部实现,性能指标全部达到要求。
    \item
      所有测试项没有残余的一级二级三级的错误。
    \item
      立项审批表、需求分析文档、设计文档和编码实现一致。
    \item
      验收测试工件齐全(测试计划,测试用例,测试日志,测试通知单,测试分析报告)
    \end{itemize}
  \end{itemize}
\end{itemize}

\hypertarget{ux8981ux6d4bux8bd5ux7684ux7279ux6027}{%
\section{要测试的特性}\label{ux8981ux6d4bux8bd5ux7684ux7279ux6027}}

\hypertarget{ux6d4bux8bd5ux7528ux4f8bux89c4ux8303}{%
\subsection{测试用例规范:}\label{ux6d4bux8bd5ux7528ux4f8bux89c4ux8303}}

\begin{itemize}
\tightlist
\item
  用例类型:功能测试、性能测试、配置相关、安装部署、安全相关、接口测试、其他
\item
  适用阶段:单元测试阶段、功能测试阶段、集成测试阶段、系统测试阶段、冒烟-
  测试阶段、版本验证阶段
\item
  相关需求:测试用例关联的需求名称
\item
  用例标题:相关需求名-用例概括-编号
\item
  优先级:1、2、3、4
\item
  前置条件:用例测试的前置条件描述
\item
  用例步骤与预期结果
\item
  关键词
\end{itemize}

\hypertarget{ux4e0dux4f1aux88abux6d4bux8bd5ux7684ux7279ux6027}{%
\section{不会被测试的特性}\label{ux4e0dux4f1aux88abux6d4bux8bd5ux7684ux7279ux6027}}

\hypertarget{ux65b9ux6cd5}{%
\section{方法}\label{ux65b9ux6cd5}}

\hypertarget{ux6682ux505cux51c6ux5219ux548cux7ee7ux7eedux51c6ux5219}{%
\section{暂停准则和继续准则}\label{ux6682ux505cux51c6ux5219ux548cux7ee7ux7eedux51c6ux5219}}

\begin{itemize}
\tightlist
\item
  暂停准则列表

  \begin{itemize}
  \tightlist
  \item
    当系统环境中必要的jar依赖包受到损坏或丢失,以至于系统无法启动时,测试暂停
  \item
    某组件相关测试准则通过率小于50\%时测试暂停
  \end{itemize}
\item
  继续准则列表

  \begin{itemize}
  \tightlist
  \item
    因测试项准则通过率过低而被暂停的测试,开发人员修复结束后继续进行
  \item
    系统能够启动(主页能够在浏览器上被正常显示)时,测试继续
  \item
    测试暂停时进行到一半的测试活动需要被重启
  \end{itemize}
\end{itemize}

\hypertarget{ux6d4bux8bd5ux4ea4ux4ed8ux7269}{%
\section{测试交付物}\label{ux6d4bux8bd5ux4ea4ux4ed8ux7269}}

\begin{itemize}
\tightlist
\item
  完整测试计划
\item
  测试设计规范
\item
  测试用例规范
\item
  测试程序规范
\item
  测试项目传送报告
\item
  测试日志
\item
  测试事故报告
\item
  测试总结报告
\item
  测试数据
\item
  测试工具记录
\end{itemize}

\hypertarget{ux6d4bux8bd5ux4efbux52a1}{%
\section{测试任务}\label{ux6d4bux8bd5ux4efbux52a1}}

\hypertarget{ux51c6ux5907ux6d4bux8bd5ux4efbux52a1}{%
\subsection{准备测试任务}\label{ux51c6ux5907ux6d4bux8bd5ux4efbux52a1}}

\hypertarget{ux6d4bux8bd5ux8ba1ux5212ux7f16ux5199}{%
\subsubsection{测试计划编写}\label{ux6d4bux8bd5ux8ba1ux5212ux7f16ux5199}}

\begin{itemize}
\tightlist
\item
  交付物:测试计划
\item
  相关依赖:详细的需求说明及用户文档
\item
  内容:对测试过程的整体规划,将成为后续任务的依赖
\end{itemize}

\hypertarget{ux6267ux884cux6d4bux8bd5ux4efbux52a1}{%
\subsection{执行测试任务}\label{ux6267ux884cux6d4bux8bd5ux4efbux52a1}}

\hypertarget{ux767dux76d2ux6d4bux8bd5}{%
\subsubsection{白盒测试}\label{ux767dux76d2ux6d4bux8bd5}}

\begin{itemize}
\tightlist
\item
  交付物:白盒测试报告
\item
  相关依赖:项目源代码及单元测试工具等
\item
  内容:通过检查软件内部的逻辑结构,对软件中的逻辑路径进行覆盖测试。在程序不同地方设立检查点,检查程序的状态,以确定实际运行状态与预期状态是否一致
\end{itemize}

\hypertarget{ux9ed1ux76d2ux6d4bux8bd5}{%
\subsubsection{黑盒测试}\label{ux9ed1ux76d2ux6d4bux8bd5}}

\begin{itemize}
\tightlist
\item
  交付物:黑盒测试报告
\item
  相关依赖:可用的各功能接口
\item
  内容:在完全不考虑程序内部结构和内部特性的情况下,在程序接口进行测试,检查程序功能是否按照需求规格说明书的规定正常使用,程序是否能适当地接收输入数据而产生正确的输出信息
\end{itemize}

\hypertarget{ux6027ux80fdux6d4bux8bd5}{%
\subsubsection{性能测试}\label{ux6027ux80fdux6d4bux8bd5}}

\begin{itemize}
\tightlist
\item
  交付物:性能测试报告
\item
  相关依赖:QALoad、LoadRunner、Benchmark
  Factory和Webstress等性能测试工具
\item
  内容:通过自动化的测试工具模拟多种正常、峰值以及异常负载条件来对系统的各项性能指标进行测试
\end{itemize}

\hypertarget{ux5b89ux5168ux6027ux6d4bux8bd5}{%
\subsubsection{安全性测试}\label{ux5b89ux5168ux6027ux6d4bux8bd5}}

\begin{itemize}
\tightlist
\item
  交付物:安全测试报告
\item
  相关依赖:国家安全标准参考
\item
  内容:验证安装在系统内的保护机制能否在实际应用中对系统进行保护,使之不被非法入侵,不受各种因素的干扰
\end{itemize}

\hypertarget{ux9759ux6001ux6d4bux8bd5}{%
\subsubsection{静态测试}\label{ux9759ux6001ux6d4bux8bd5}}

\begin{itemize}
\tightlist
\item
  交付物:静态测试报告
\item
  相关依赖:项目源代码,代码书写规范,各阶段文档
\item
  内容:通过分析或检查源程序的语法、结构、过程、接口等来检查程序的正确性。对需求规格说明书、软件设计说明书、源程序做结构分析、流程图分析、符号执行来寻找漏洞
\end{itemize}

\hypertarget{ux573aux666fux6d4bux8bd5}{%
\subsubsection{场景测试}\label{ux573aux666fux6d4bux8bd5}}

\begin{itemize}
\tightlist
\item
  交付物:场景测试报告
\item
  相关依赖:能够顺利工作的完整系统
\item
  内容:模拟特定场景边界发生的事情,通过事件来触发某个动作的发生,观察事件的最终结果,从而用来发现需求中存在的问题
\end{itemize}

\hypertarget{ux73afux5883ux9700ux6c42}{%
\section{环境需求}\label{ux73afux5883ux9700ux6c42}}

\begin{itemize}
\tightlist
\item
  硬件环境

  \begin{itemize}
  \tightlist
  \item
    能够连到互联网的个人电脑(若为Windows则需Windows7以上系统,内存大于等于2G)
  \item
    方便的实时语音通讯工具
  \end{itemize}
\item
  软件环境

  \begin{itemize}
  \tightlist
  \item
    Java 1.8 运行环境
  \item
    Maven
  \item
    适当的Java IDE
  \end{itemize}
\item
  测试数据

  \begin{itemize}
  \tightlist
  \item
    用户数据包含多位主持人、作者、评审员、质管员,具体数据来源复旦大学软件学院教师资料
  \end{itemize}
\item
  工具

  \begin{itemize}
  \tightlist
  \item
    \url{https://codecov.io}
  \item
    \url{https://jenkins.io}
  \item
    \url{https://semmle.com/codeql}
  \item
    \url{https://travis-ci.org}
  \item
    \url{https://www.appveyor.com}
  \end{itemize}
\end{itemize}

\hypertarget{ux8d23ux4efb}{%
\section{责任}\label{ux8d23ux4efb}}

\begin{itemize}
\tightlist
\item
  出题系统测试小组:

  \begin{itemize}
  \tightlist
  \item
    Chen
    Wang:管理与出题系统功能相关的所有测试、检查测试用例设计是否符合规范、总结测试结果
  \item
    Jiaxing Liu:为小组负责的测试项设计测试用例、作证测试用例执行结果
  \item
    Chenning Wu:执行测试用例并记录
  \end{itemize}
\item
  题库管理系统测试小组:

  \begin{itemize}
  \tightlist
  \item
    Yihan
    Zhang:管理与题库管理系统功能相关的所有测试、检查测试用例设计是否符合规范、总结测试结果
  \item
    Cenyuan Zhang:为小组负责的测试项设计测试用例、作证测试用例执行结果
  \item
    Ruixin Shi:执行测试用例并记录
  \end{itemize}
\item
  集成测试小组:

  \begin{itemize}
  \tightlist
  \item
    Jia'ni
    Huang:管理与系统非功能需求相关的所有测试及出题系统与题库管理系统之间的接口测试、检查测试用例设计是否符合规范、总结测试结果
  \item
    Jiahe Tian:为小组负责的测试项设计测试用例、作证测试用例执行结果
  \item
    Xiaoxiao Hu:执行测试用例并记录
  \end{itemize}
\item
  提供测试项的小组:Ruixin Shi、Xiaoxiao Hu、Jiahe Tian
\item
  提供环境需求的小组:Chen Wang、Jiaxing Liu、Cenyuan Zhang
\end{itemize}

\hypertarget{ux4ebaux624bux548cux57f9ux8badux7684ux9700ux8981}{%
\section{人手和培训的需要}\label{ux4ebaux624bux548cux57f9ux8badux7684ux9700ux8981}}

\begin{itemize}
\tightlist
\item
  测试工具培训

  \begin{itemize}
  \tightlist
  \item
    培训师:王宸
  \item
    资质:参与多次大型项目开发与测试工作,熟悉当今流行的测试工具
  \end{itemize}
\item
  测试流程规范培训

  \begin{itemize}
  \tightlist
  \item
    培训师:刘佳兴
  \item
    资质:上市公司合作人,具有丰富的管理经验,可以轻松协调安排组内成员测试的规范性和一致性
  \end{itemize}
\end{itemize}

\begin{tabular}{|p{2cm}|p{3.3cm}|p{8cm}|}
\hline
测试人员 & 需求技能 & 需求培训\\
\hline
Chen Wang & 熟悉测试工具、文档规范 & 测试流程规范培训\\
\hline
Jiaxing Liu & 熟悉测试用例设计规范 & 测试工具培训\\
\hline
Chenning Wu & 熟悉文档规范、程序代码、测试工具 & 测试流程规范培训、测试工具培训\\
\hline
Yihan Zhang & 熟悉测试工具、文档规范 & 测试流程规范培训、测试工具培训\\
\hline
Cenyuan Zhang & 熟悉测试用例设计规范 & 测试流程规范培训、测试工具培训\\
\hline
Ruixin Shi & 熟悉文档规范、程序代码、测试工具 & 测试流程规范培训、测试工具培训\\
\hline
Jia'ni Huang & 熟悉测试工具、文档规范 & 测试流程规范培训、测试工具培训\\
\hline
Jiahe Tian & 熟悉测试用例设计规范 & 测试流程规范培训、测试工具培训\\
\hline
Xiaoxiao Hu & 熟悉文档规范、程序代码、测试工具 & 测试流程规范培训、测试工具培训\\
\hline
\end{tabular}

\hypertarget{ux65f6ux95f4ux8868}{%
\section{时间表}\label{ux65f6ux95f4ux8868}}

\hypertarget{ux603bux4f53ux65f6ux95f4ux8868}{%
\subsection{总体时间表}\label{ux603bux4f53ux65f6ux95f4ux8868}}

\begin{longtable}[]{@{}lll@{}}
\toprule
\begin{minipage}[b]{0.13\columnwidth}\raggedright
时间\strut
\end{minipage} & \begin{minipage}[b]{0.28\columnwidth}\raggedright
测试任务内容\strut
\end{minipage} & \begin{minipage}[b]{0.50\columnwidth}\raggedright
说明\strut
\end{minipage}\tabularnewline
\midrule
\endhead
\begin{minipage}[t]{0.13\columnwidth}\raggedright
3月24日-4月5日\strut
\end{minipage} & \begin{minipage}[t]{0.28\columnwidth}\raggedright
测试计划的编撰\strut
\end{minipage} & \begin{minipage}[t]{0.50\columnwidth}\raggedright
-\strut
\end{minipage}\tabularnewline
\begin{minipage}[t]{0.13\columnwidth}\raggedright
4月6日-4月20日\strut
\end{minipage} & \begin{minipage}[t]{0.28\columnwidth}\raggedright
白盒测试用例设计与执行\strut
\end{minipage} & \begin{minipage}[t]{0.50\columnwidth}\raggedright
4月6日-4月10日:小组讨论及用例编写 4月11日-4月18日:具体执行及结果记录
4月19日-4月20日:总结及报告撰写\strut
\end{minipage}\tabularnewline
\begin{minipage}[t]{0.13\columnwidth}\raggedright
4月21日-5月11日\strut
\end{minipage} & \begin{minipage}[t]{0.28\columnwidth}\raggedright
黑盒测试用例设计与执行\strut
\end{minipage} & \begin{minipage}[t]{0.50\columnwidth}\raggedright
4月21日-4月26日:小组讨论及用例编写 4月27日-5月9日:具体执行及结果记录
5月10日-5月11日:总结及报告撰写\strut
\end{minipage}\tabularnewline
\begin{minipage}[t]{0.13\columnwidth}\raggedright
5月12日-5月18日\strut
\end{minipage} & \begin{minipage}[t]{0.28\columnwidth}\raggedright
性能测试用例设计与执行\strut
\end{minipage} & \begin{minipage}[t]{0.50\columnwidth}\raggedright
5月12日-5月13日:小组讨论及用例编写 5月14日-5月16日:具体执行及结果记录
5月17日-5月18日:总结及报告撰写\strut
\end{minipage}\tabularnewline
\begin{minipage}[t]{0.13\columnwidth}\raggedright
5月19日-5月25日\strut
\end{minipage} & \begin{minipage}[t]{0.28\columnwidth}\raggedright
安全性测试与静态测试用例设计与执行\strut
\end{minipage} & \begin{minipage}[t]{0.50\columnwidth}\raggedright
5月19日-5月20日:小组讨论及用例编写 5月21日-5月23日:具体执行及结果记录
5月24日-5月25日:总结及报告撰写\strut
\end{minipage}\tabularnewline
\begin{minipage}[t]{0.13\columnwidth}\raggedright
5月26日-6月1日\strut
\end{minipage} & \begin{minipage}[t]{0.28\columnwidth}\raggedright
场景测试用例设计与执行\strut
\end{minipage} & \begin{minipage}[t]{0.50\columnwidth}\raggedright
5月26日-5月27日:小组讨论及用例编写 5月28日-5月30日:具体执行及结果记录
5月31日-6月1日:总结及报告撰写\strut
\end{minipage}\tabularnewline
\bottomrule
\end{longtable}

\hypertarget{ux91ccux7a0bux7891}{%
\subsection{里程碑}\label{ux91ccux7a0bux7891}}

\begin{itemize}
\tightlist
\item
  需求冻结:需求完全确定且不再进行版本更新
\item
  计划完成:测试计划书定稿
\item
  白盒测试完成:所有测试用例通过率达到100\%,代码覆盖率达80\%以上
\item
  黑盒测试完成:所有测试用例通过率达到100\%
\item
  安全性测试完成:所有测试用例通过率达到100\%
\item
  静态测试完成:代码全部审核完成且更改完毕
\item
  场景测试完成:所有测试用例通过率达到100\%
\item
  项目发布
\end{itemize}

\hypertarget{ux98ceux9669ux4ee5ux53caux5e94ux6025ux63aaux65bd}{%
\section{风险以及应急措施}\label{ux98ceux9669ux4ee5ux53caux5e94ux6025ux63aaux65bd}}

\hypertarget{ux6388ux6743}{%
\section{授权}\label{ux6388ux6743}}

\begin{itemize}
\tightlist
\item
  核准计划负责人:沈立炜(总监)
\item
  签名:
\item
  日期:
\end{itemize}

\pagebreak

\hypertarget{ux53c2ux8003ux6587ux732e}{%
\section*{参考文献}\label{ux53c2ux8003ux6587ux732e}}
\addcontentsline{toc}{section}{参考文献}

\hypertarget{refs}{}
\leavevmode\hypertarget{ref-innovativeInternationalisation}{}%
International Organization for Standardization. 2014. \emph{Systems and
Software Engineering --- Systems and Software Quality Requirements and
Evaluation (SQuaRE) --- Guide to SQuaRE}. \emph{International
Organization for Standardization}. Vol. 2014.
\url{https://www.iso.org/standard/64764.html}.

\leavevmode\hypertarget{ref-innovative1}{}%
中国国家标准化管理委员会. 2016. \emph{GB/T
25000.51-2016《系统与软件工程系统与软件质量要求和评价 (SQuaRE) 第 51
部分 : 就绪可用软件产品 (RUSP) 的质量要求和测试细则》}.
\emph{系统与软件工程系统与软件质量要求和评价 (SQuaRE)}. Vol. 51.
中国国家标准化管理委员会. \url{http://openstd.samr.gov.cn}.

\leavevmode\hypertarget{ref-innovative3}{}%
---------. 2017a. \emph{GB/T 25000.12-2017《系统与软件工程
系统与软件质量要求和评价(SQuaRE) 第12部分:数据质量模型》}.
\emph{系统与软件工程系统与软件质量要求和评价 (SQuaRE)}. Vol. 12.
中国国家标准化管理委员会. \url{http://openstd.samr.gov.cn}.

\leavevmode\hypertarget{ref-innovative4}{}%
---------. 2017b. \emph{GB/T 25000.24-2017《系统与软件工程
系统与软件质量要求和评价(SQuaRE) 第24部分:数据质量测量》}.
\emph{系统与软件工程系统与软件质量要求和评价 (SQuaRE)}. Vol. 24.
中国国家标准化管理委员会. \url{http://openstd.samr.gov.cn}.

\leavevmode\hypertarget{ref-innovative5}{}%
---------. 2018. \emph{GB/T 25000.40-201《系统与软件工程
系统与软件质量要求和评价(SQuaRE) 第40部分:评价过程》}.
\emph{系统与软件工程系统与软件质量要求和评价 (SQuaRE)}. Vol. 40.
中国国家标准化管理委员会. \url{http://openstd.samr.gov.cn}.

\leavevmode\hypertarget{ref-innovative2}{}%
---------. 2019. \emph{GB/T 25000.23-2019《系统与软件工程
系统与软件质量要求和评价(SQuaRE) 第23部分:系统与软件产品质量测量》}.
\emph{系统与软件工程系统与软件质量要求和评价 (SQuaRE)}. Vol. 23.
中国国家标准化管理委员会. \url{http://openstd.samr.gov.cn}.

\end{document}
